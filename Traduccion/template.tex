%%%%%%%%%%%%%%%%%%%%%%% file template.tex %%%%%%%%%%%%%%%%%%%%%%%%%
%
% This is a template file for the LaTeX package SVJour2 for the
% Springer journal "Biological Cybernetics"
%
%                                    Springer Heidelberg 2004/11/22
%
% Copy it to a new file with a new name and use it as the basis
% for your article
%
%%%%%%%%%%%%%%%%%%%%%%%%%%%%%%%%%%%%%%%%%%%%%%%%%%%%%%%%%%%%%%%%%%%
%
% First comes an example EPS file -- just ignore it and
% proceed on the \documentclass line
% your LaTeX will extract the file if required
\begin{filecontents*}{example.eps}
%!PS-Adobe-3.0 EPSF-3.0
n%%BoundingBox: 19 19 221 221
%%CreationDate: Mon Sep 29 1997
%%Creator: programmed by hand (JK)
%%EndComments
gsave
newpath
  20 20 moveto
  20 220 lineto
  220 220 lineto
  220 20 lineto
closepath
2 setlinewidth
gsave
  .4 setgray fill
grestore
stroke
grestore
\end{filecontents*}
%
\documentclass[twocolumn,fleqn]{svjour3}
% Biological Cybernetics uses author year
% references, hence the natbib package is activated - use
% \citet{...} and \citep{...} with it to cite references.
%
\smartqed  % flush right qed marks, e.g. at end of proof
%
\usepackage[numbers]{natbib}
\usepackage{amssymb}
\usepackage{amsmath}
\usepackage{textcomp}
\usepackage[T1]{fontenc}
\usepackage[utf8]{inputenc}
\usepackage[english, spanish]{babel}
\usepackage{csquotes}
\usepackage{enumerate}
\usepackage{enumitem}
\usepackage{caption}
\captionsetup{compatibility=false}
\usepackage{subcaption}
\usepackage{listings}
% para la lista de simbolos
\usepackage{array} %for vertical thick lines in tables
\usepackage{multirow} %multirow tables
\usepackage{nicefrac} %for fractions like 1/4
\usepackage[dvipsnames]{xcolor}
\usepackage{pgfplots}
\pgfplotsset{compat=default}
\usepackage{pgfplotstable}
\usepgfplotslibrary{statistics}
\usetikzlibrary{spy}
\usepackage{longtable}
\usepackage{array}
\usepackage{pdflscape}
\usepackage{booktabs}
\usepackage{graphicx} % figuras
%\usepackage{subfigure} % subfiguras



%
% \usepackage{mathptmx}      % use Times fonts if available on your TeX system
%
% insert here the call for the packages your document requires
%\usepackage{latexsym}
% etc.
%
% please place your own definitions here and don't use \def but
% \newcommand{}{}
%
\journalname{}

\DeclareUnicodeCharacter{2217}{*}

\begin{document}

%\title{Ordenamiento de colores RGB basado en m\'etricas asociadas a la imagen}
\title{RBG ordering based on image metrics}
\author{
  Jos\'e Luis V\'azquez Noguera \and
  Christian E. Schaerer \and
  Jacques Facon	\and
  Horacio Legal Ayala
}

\institute{
  Jos\'e Luis V\'azquez Noguera  $\cdot$ Horacio Legal Ayala $\cdot$ Christian E. Schaerer \at  
  Polytechnic Faculty, National University of Asuncion - San Lorenzo, Paraguay \\
  \email{\{jlvazquez,hlegal,cschaer\}@pol.una.py}  
  \and
  Jacques Facon \at
  PPGIa - PUCPR-Pontifícia Universidade Catolica do Paraná - Curitiba - Pr, Brazil \\
  \email{facon@ppgia.pucpr.br}  
}

\date{Received: date / Revised: date}
% The correct dates will be entered by the editor
\maketitle

\begin{abstract}
El orden lexicogr\'afico y sus variantes son los m\'as utilizados en la literatura para el ordenamiento de colores.  Un problema usual de este tipo de ordenamiento es el establececimiento a priori del componente de color m\'as importante y el resultado de las comparaciones lexicogr\'aficas casi siempre se deciden en los primeros componentes.  
%La norma o la distancia a un color de referencia es tambi\'en bastante utilizado como estrategia de ordenamiento, pero en las im\'agenes a color RGB dos colores visuales distintos pueden tener la misma norma, o distancia a un color de referencia. Debido a que no existe un orden natural entre los colores RGB ser\'ia de utilidad encontrar una estrategia de ordenamiento que sea dependiente de la imagen y de la aplicaci\'on. 
En este art\'iculo se propone un  ordenamiento de colores RGB, en el cu\'al se asigna una ponderaci\'on a cada componente de color (R,G o B) de acuerdo a m\'etricas aplicadas a cada componente. Con esto se busca que la prioridad de los componentes de color sean dictaminados por información extraida de la propia imagen y no sea definida de manera arbitraria. Las aplicaciones utilizadas para la validación de la propuesta: son el filtrado de im\'agenes, mejora de contraste y caracterización texturas para su posterior clasificación. Los resultados utilizando el ordenamiento propuesto  en las diferentes aplicaciones son mejores en la mayor\'ia de los casos en comparaci\'on a diferentes m\'etodos de ordenamiento del estado del arte. 
%Las m\'etricas asociadas a cada componente con buenos resultados varian de acuerdo a la aplicaci\'on seleccionada.    
\end{abstract}

\begin{abstract}
  In color ordering literature, the lexicographical ordering and its variants are the most used methods and an usual problem is the ``a priori'' definition of the most important color component \textbf{(NE...)}
  This paper proposes a new ordering criteria, which assign a weight to every component in accordance with the metrics applied to it, with this criteria we pursue to avoid arbitrary definitions and build the ordering based on image-based information. \textbf{... Obtener los terminos correctos de fuentes oficiales (filtrado de imagenes, mejora de constraste...)}
\end{abstract}

%\section{Introducci\'on}
\section{Introduction}
\label{intro}
El procesamiento digital de im\'agenes a color tiene semejanza con la visi\'on humana \cite{roerdink2000watershed}, que es crom\'atica, y basa su importancia por el acrecentamiento de la informaci\'on que aporta al an\'alisis de las im\'agenes, en contrapartida con las im\'agenes en escala de grises que aportan menos informaci\'on al trabajar solo con intensidades o im\'agenes binarias que pueden tener solo dos valores posibles, blanco o negro. En sus inicios, los algoritmos de procesamiento digital de im\'agenes fueron desarrollados para im\'agenes binarias o im\'agenes en escala de grises. Durante bastante tiempo solo se trabajaba con estos dos tipos de im\'agenes debido a la limitaci\'on de la infraestructura computacional, ya que los elevados tiempos de c\'omputo de los algoritmos de procesamiento digital de im\'agenes obligaba a reducir la informaci\'on visual a solo un plano bidimensional \cite{ortiz2002procesamiento}.

Digital image processing on color images resemble human vision \cite{roerdink2000watershed}, which is \textbf{NE...}, on the other hand grayscale images and binary images contribute less information because of their use of single dimentional intensities and binary values, white and black. At its inception, digital image processing algorithms where only develop for grayscale and binary images because of the limited computing power available at the time and their high cost, these conditions demands to reduce the visual information into a single plane.
\textbf{ver citas en inglés...}

Informaci\'on importante puede ser distinguida en im\'agenes en escala de grises, como los bordes que se dan en los lugares que existen cambios bruscos de niveles de intensidades. Por medio del c\'alculo del gradiente se puede extraer los bordes y de esa manera obtener los contornos de los objetos que lo separan del fondo. En ocasiones, los reflejos en las im\'agenes afectan la intensidad luminosa de los objetos produciendo errores en la detecci\'on de las fronteras o contornos de los mismos. Estos efectos de la iluminaci\'on, reflejos, y la perdida de informaci\'on crom\'atica, hacen que muchos algoritmos de procesamiento de im\'agenes en escala  de grises no sean tan eficientes \cite{ortiz2002procesamiento}. Bajo esta perspectiva, 	y con el avance actual de los recursos o infraestructuras computacionales, con procesadores destinados a algoritmos de procesamiento digital de im\'agenes, muchos algoritmos de im\'agenes en escala de grises se est\'an extendiendo a im\'agenes a color, aprovechando la mayor cantidad de informaci\'on que puede brindar de una escena capturada \cite{ortiz2002procesamiento}.

Valuable information can be obtain from grayscale images, for example an object boundaries could been detected from sudden changes in intensities. By calculating the gradient \textbf{on every pixel (Added)} we could extract every object from the image, however unexpected reflections could produce errors on the boundaries detections. Reflections, lighting effects and the lack of chromatic information limit the efficiency of many grayscale algorithms \cite{ortiz2002procesamiento}. Considering these ideas and the new improvements on computational resources, as image processing specialized processors \textbf{review}, a lot of grayscale images are been generalized into color images \cite{ortiz2002procesamiento}.
  
Los espacios de color son formalismos que permiten la definici\'on de colores, y establecen propiedades para su manipulaci\'on \cite{joblove1978color,meyer1980perceptual}.

Color spaces are formalisms \textbf{(alternative)} that allow the definition of color and their properties for proper manipulation \cite{joblove1978colo,meyer1980perceptual}.

El espacio de color m\'as conocido y comunmente utilizado por los monitores es el RGB, que est\'a cimentado en el modelo triest\'imulo y s\'intesis aditiva de color \cite{busin2008color}. En el espacio de color RGB los colores son representados como vectores de 3 componentes, el rojo, el verde y el azul. La cantidad asociada a cada componente indica cu\'anto interviene dicho color primario para la mezcla y representaci\'on del color \cite{tkalcic2003colour}. En el espacio de color CMY los colores cyan, amarillo y magenta  representan la s\'intesis sustractiva de color \cite{rolleston1996color}.  Estos colores son conocidos como colores secundarios. El espacio de color CMYK est\'a representado por 4 componentes, donde el componente K (componente de tinta negra) representa el valor m\'aximo entre los 3 colores secundarios \cite{tkalcic2003colour}. Las impresoras utilizan este espacio de color \cite{rolleston1996color}.

The most used color space on screens is the RGB color space, which it is based on the tristimulus model and aditive syntesis of color \textbf{review from cite}. In the RBG color space a color, where the colors used are red, green and blue, is defined as 3-tuples where the scalar value on every component measures its influence on the mix \cite{tkalcic2003colour}. In the CMY color space, cyan, magenta and yellow also known as secondary colors, represents the substractive syntesis of color \cite{rolleston1996color}. The CMYK color space, used in printers\cite{rolleston1996color}, extends the previous color space adding the K component which represents the maximum value between the 3 secondary values\cite{tkalcic2003colour} \textbf{NE}.

A causa de que ciertos colores solo pueden representarse con un valor negativo de est\'imulo fue introducido el espacio de color XYZ, que es obtenido por una trasformaci\'on lineal del espacio de color RGB \cite{ortiz2002procesamiento}. El espacio de color XYZ se utiliza cuando la representaci\'on del color es independiente del hardware.

The XYZ color space has been introduced because there are some colors that can only be represented by a negative value of stimuli \textbf{review} and it is a linear transformation over the RGB color space\cite{ortiz2002procesamiento}. This color space is used when the color representation is independent of the hardware.

El espacio de color L*a*b* es un espacio tridimensional, en donde L* representa la luminosidad de negro a blanco, a* codifica la sensaci\'on rojo-verde, y b* codifica la sensaci\'on amarillo-azul \cite{leon2006color}. 
Los espacios de color CIELAB y CIELUV representan el color de manera que sea uniformente lineal, es decir, un cambio de color debe producir el mismo cambio o importancia visual \cite{mahy1994evaluation}. Se utiliza para aplicaciones industriales, donde se busca medir el color de los objetos.  Por otra parte est\'an los espacios de color utilizados en la radifusi\'on de la se\~nal de televisi\'on, estos son el YIQ, y el YUV \cite{munson1995color}.

The L*a*b* is a 3 component color space where the L* represents the luminosity from black to white, the a* measures from red to green and the b* from yellow to blue\cite{leon2006color} \textbf{review from cites}.

Por \'ultimo podemos mencionar los espacios de color HSI, HLS, HSV y sus variantes que son los que m\'as se asemejan a la visi\'on humana, por tener en cuenta los atributos de percepci\'on de luminancia, saturaci\'on y matiz \cite{zamora2001comparative}.

The color spaces HSI, HLS, HSV and their variants are the color spaces that resembles the human vision the most because they are based on luminosity, saturation and hue perseptions\cite{zamora2001comparativos}.

%Los filtros de orden son operaciones de vecindad no lineal que se utilizan para diversas aplicaciones en im\'agenes en escala de grises \cite{pitas1992order}. 
Muchas aplicaciones como la utilización de filtros para eliminaci\'on de ruido, estiramiento de contraste, detecci\'on de bordes y segmentaci\'onen imágenes de color necesitan de ordenamiento de colores  \cite{ortiz2002procesamiento}.  Debido a que las im\'agenes a color son representados por vectores \textit{multi}-dimensionales y que no existe un orden natural para los mismos, la extensi\'on de los filtros de orden para im\'agenes a color no es trivial.

A lot of aplications needs color ordering, as noise reduction, contrast enhancement, borders detection and segmentation of color images\textbf{review terminos}\cite{ortiz2002procesamiento}. Because of \textit{multi}-dimentional nature of color representation their is no natural order and \textbf{NE}.

En este trabajo se establece una nuevo ordenamiento de colores RGB basada en m\'etricas asociadas a cada  componente de la imagen.  El ordenamiento propuesto es comparada con diferentes m\'etodos de orden del estado del arte en las aplicaciones de eliminaci\'on de diferentes ruidos, estiramiento de contraste y caracterización de texturas para poder clasificarlas.

This work presents a new ordering for the RBG color space based on the metrics associated to each component. The proposed ordering is compared with the state of art orderings on noise reduction, contrast enhancement and texture characterization. \textbf{IMP REVIEW}

El resto del art\'iculo est\'a organizado de la siguiente manera. En la Secci\'on 2 se presenta el estado del arte. En la Secci\'on 3 se presenta los fundamentos de filtrado de imagen a color, donde se explican	los conceptos de filtrado de im\'agenes, ordenamiento y matem\'atica morfol\'ogica. En la Secci\'on 4 se presenta el  ordenamiento propuesto. En la Secci\'on 5 se pueden observar los resultados experimentales del ordenamiento propuesto en comparaci\'on con las del estado del arte en las aplicaciones de  eliminaci\'on de ruido, estiramiento de contraste y caracterización de texturas para su posterior clasificaci\'on. Por \'ultimo en la Secci\'on 6 se presentan las conclusiones junto a los trabajos futuros.

The paper is organized as follows. The second section presents the current state of the art on the matter. The third section presents the fundamentas of color image filtering as main conceps of image filtering, ordering and mathematics morphology. The fourth section presents the proposed ordering. The fifth section presents the experimental results of the comparisons with the current state of the art on aplications as noise reduction, contrast enhancement and texture characterization. Finally, the sixth and last section presents our conclutions for future works.\textbf{review}.

\section{State of the art \textbf{methods.??}}
\label{Relacionados}

La extensi\'on de los filtros de orden a im\'agenes a color requiere, por una parte seleccionar el espacio color en el que se procesa la imagen y por otra, establecer un orden en \'este espacio de color. Para establecer un ordenamiento se han trabajado en diferentes espacios de color, entre los que podemos citar, los espacios de color L*a*b*\cite{hanbury2001mathematical}, HLS \cite{hanbury2001mathematical2},  CIELAB \cite{hanbury2002mathematical3}, HSI \cite{tobar2007mathematical}, HSV \cite{lei2013vector} y el espacio de color RGB  \cite{zaharescu2003color, gao2013adaptive, wang2012edge}.

\textbf{....}

La matemática morfológica nació en 1964 de la colaboración de Georges Matheron y Jean Serra en la École des Mines de Paris, Francia \cite{serra1982image}. Actualmente, el ámbito y alcance de los procesamientos morfológicos es tan amplio como el propio procesamiento de imágenes. Se pueden encontrar aplicaciones tales como la segmentación, restauración, detección de bordes, aumento de contraste, análisis de texturas, compresión, entre otros \cite{ortiz2002procesamiento}. 
La erosi\'on y la dilataci\'on son las operaciones b\'asicas de la matem\'atica morfol\'ogica, donde se busca establecer un ret\'iculo completo \cite{heijmans1990algebraic}. La erosi\'on es el m\'inimo y la dilataci\'on es el m\'aximo dentro de una ventana llamada elemento estructurante. A partir de estas dos operaciones b\'asicas se extiende toda la matem\'atica morfol\'ogica.  Para extender la matem\'atica morfol\'ogica a color es necesario establecer un orden, de manera a poder encontrar el m\'inimo y el m\'aximo dentro del elemento estructurante.

Mathematical morphology borns in 1964 from the colaboration of Georges Matheron and Jean Serra at École de Mines, Paris \cite{serra1982image} and currently ir reach is as broad as image processing itself. A few examples of aplications are image segmentation, restoration, border detections, contrast enhacement, texture analysis, compression, etc. \cite{ortiz2002procesamiento}.
Erosion and dilation are the most basic operations where \textbf{NE reticulo?}\cite{heijmans1990algebraic}. Erosion is the minimum and dilation is the maximum of a window \textbf{review for better terminology} called structural element and from this two all other operations are builded. To generalize the mathematical morphology to color images is required an ordering between colors to be able to determine a minimum and a maximum inside the structural element.

Varias publicaciones recientes han propuesto extensiones de matemática morfológica a imagenes de color \cite{ledoux2012limits, van2013group, velasco2012random, lezoray2009learning, velasco2010morphological, burgeth2013morphology, velasco2011supervised, hanbury2001morphological, angulo2010pseudo, aptoula2008alpha, kleefeld2015processing, vazquez2014color}. Espec\'ificamente para el espacio de color RGB el ordenamiento mediante el entrelazado de bits se ha mostrado eficiente para el filtrado de im\'agenes a color \cite{chanussot1997bit}. Para una revisión más detallada sobre métodos de morfología matemática a color, sugerimos ver el artículo de Aptoula y Lefevre \cite{aptoula2007comparative}.

Recent publications presented generalizations of mathematical morphology \cite{ledoux2012limits, van2013group, velasco2012random, lezoray2009learning, velasco2010morphological, burgeth2013morphology, velasco2011supervised, hanbury2001morphological, angulo2010pseudo, aptoula2008alpha, kleefeld2015processing, vazquez2014color}. In RGB the interlace bit ordering \textbf{review...!!} has been proved to be efficient on color image filtering \cite{chanussot1997bit}. For a more in depth analysis on mathematical morphology methods, we suggest Aptoula and Lefevre \cite{apatoula2007comparative}.

%El art\'iculo \cite{aptoula2007comparative} incluye m\'as de 70 referencias distintas de m\'etodos de morfolog\'ia matem\'atica a color, mostrando as\'i, aparte de una revisi\'on del estado del arte, que el \'area es reciente. Se han propuesto una gran cantidad de m\'etodos para realizar la extensi\'on de morfolog\'ia matem\'atica a color; entre los art\'iculos recientes podemos citar \cite{ledoux2012limits, van2013group, velasco2012random, lezoray2009learning, velasco2010morphological, burgeth2013morphology, velasco2011supervised, hanbury2001morphological, angulo2010pseudo, aptoula2008alpha, kleefeld2015processing, vazquez2014color}. 


De manera general, es decir para muchos espacios de color, el ordenamiento lexicogr\'afico es  uno de los m\'as utilizados en la literatura \cite{aptoula2007comparative, aptoula2008lexicographical}, ya que posee propiedades te\'oricas deseables y permite personalizar f\'acilmente la manera que se van a comparar los componentes de la imagen. 

Louverdis et. al. \cite{louverdis2002new} y  Vardavoulia et. al. \cite{vardavoulia2002vector} presentan un ordenamiento lexicográfico en el espacio de color HSV para el procesamiento de morfológico de imágenes a color, mientras que Louverdis et. al \cite{louverdis2002morphological} utiliza el mismo orden y espacio de color para presentar una nueva técnica de morfológica para el analisis de forma y tamaño de imágenes granulares. Angulo y Serra \cite{angulo2003morphological} discuten el uso del ordenamiento lexicográfico en los espacios de color RGB y HLS para la compresión imágenes a color JPEG. Ortiz et. al. \cite{ortiz2004gaussian}  utilizan el ordenamiento lexicográfico I→H→S $(Href=0º)$ para la eliminación de ruido gaussiano. 

Louverdis et. al. \cite{louverdis2002new} and Vardavoulia et. al. \cite{vardavoulia2002vector} presents a lexicographic ordering in HSV, while Louverdis et. al. \cite{louverdis2002morphological} uses the same ordering and color space to develop a novel morphologic method for shape and size analisys on granular images. Angulo and Serra \cite{angulo2003morphological} discuss over lexicographic ordering on RGB and HLS for JPEG image compression. Ortiz et. al. \cite{ortiz2004morphological} uses the I→H→S $(Href=0º)$ lexicographical ordering for gaussian noise elimination.

%Un ejemplo de ordenamiento lexicogr\'afico en el espacio de color HSV se puede encontrar en \cite{louverdis2002new}, mientras que \cite{louverdis2002morphological} y } utilizan el mismo orden, respectivamente, para el filtro de la mediana y el c\'alculo de granulometr\'ia en im\'agenes a color. Por supuesto, tambi\'en puede haber situaciones espec\'ificas donde la informaci\'on crom\'atica es m\'as significativa, por ejemplo, en \cite{ortiz2002colour} la matiz se compara por primera vez en la cascada lexicogr\'afica.  En \cite{ortiz2004gaussian} la eliminaci\'on de ruido se consigue por medio del ordenamiento lexicogr\'afico en el espacio HSI, utilizando la intensidad en la primera posici\'on en la cascada lexicogr\'afica. El espacio L*a*b* \cite{hanbury2002mathematical3} ha sido tambi\'en utilizado con un ordenamiento lexicogr\'afico. Por otra parte, un estudio a fondo del potencial de este orden en el espacio HLS se proporciona en \cite{hanbury2001mathematical2}, mientras que en \cite{angulo2003mathematical} el uso del ordenamiento lexicogr\'afico en el espacio HLS mejorado (IHLS) ha sido explorado.

El ordenamiento lexicogr\'afico sufre de un serio inconveniente. M\'as precisamente, el resultado de la gran mayor\'ia de las comparaciones lexicogr\'aficas, se decide casi siempre en los primeros componentes del vector que se comparan, mientras que la contribuci\'on de las dimensiones restantes puede considerarse insignificante \cite{hanbury2002mathematical3}. 

The lexicographical ordering suffers from a serious inconvinient. More precisely, the final result of most lexicographical ordering are highly biased towards the first components where the last components are virtually ignored \cite{hanbury2002mathematical3} \textbf{over reaction?}.
 
Con el fin de mejorar la sintonizaci\'on del grado de influencia de cada componente del vector en el resultado de comparaci\'on fueron propuestos variaciones del ordenamiento lexicogr\'afico. Un grupo de variantes es basado en el uso de un componente adicional durante la comparaci\'on.



Angulo \cite{angulo2005morphological} y  Sartor et. al. \cite{sartor2001morphological} ubican en la primera posici\'on de la cascada lexicogr\'afica una medida de distancia a un vector de referencia. Comer et. al. \cite{comer1999morphological} emplean la norma euclidea como método de ordenación de pixeles, es decir el píxel de referencia es el color negro (0,0,0) en RGB. Dos colores RGB pueden ser pr\'acticamente iguales visualmente pero ser diferente en valor de norma, o distancia a un color de referencia, as\'i como dos colores distintos tener la misma norma, por lo que no es recomendable utilizar esta estrategia. En el espacio L*a*b existe una medida de distancia definida, respecto al origen $L=0$, $a=0$ y $b=0$ que es muy utilizada para evaluación de la calidad de la reproducción en color, o en técnicas de comprensión de imágenes a color \cite{tremeau1998analyse}.   

Otros tipos de ordenamiento que buscan la extensi\'on del ordenamiento lexicogr\'afico consiste en utilizar un par\'ametro $\alpha$ definido por el usuario de manera a modificar el grado de influencia del primer componente \cite{ortiz2002procesamiento,angulo2005unified}.  A\'un con las variaciones del ordenamiento lexicogr\'afico, los criterios de la elecci\'on de cu\'al componente tendr\'a mayor prioridad en la comparaci\'on, y del valor $\alpha$ son en su mayoria arbitrarios. Gao et. al. \cite{gao2013adaptive} trata de solucionar este problema presentando  un enfoque de ordenamiento lexicogr\'afico adaptativo. 
Con el objetivo de evitar al m\'aximo la intervenci\'on subjetiva del usuario, ser\'ia de gran
importancia que los criterios arbitrarios del orden lexicogr\'afico y sus variantes puedan ser eliminados o disminuidos.


Bouchet et. al. \cite{bouchet2016fuzzy} utiliza l\'ogica difusa de manera que los 3 componentes de color tengan la misma ponderaci\'on en el ordenamiento, aunque es deseable que la prioridad de los componentes del vector que representa la imagen esten dictaminados por informaci\'on propia de la imagen, no siendo exactamente igual en todos los casos. Benavent et. al. \cite{benavent2012mathematical} presenta un m\'etodo de ordenamiento que es dependiente de la imagen y que ordena los colores de acuerdo a la densidad de probabilidad de la aparici\'on de colores de la imagen. 

La diferencia principal de esta propuesta, con los presentados en el estado del arte, radica en la extracci\'on de informaci\'on de cada componente del color RGB en un dominio espec\'ifico de la imagen. Esta informaci\'on se extrae por medio de un vector de pesos, que son calculados previamente por una funci\'on aplicada a cada uno de los componentes del color RGB. 



\section{Fundamentos de filtrado de imagen a color}
\label{Teo}

En esta sección se presentan las formulaciones formales de los conceptos teóricos detrás de la extensión de los filtros de orden para imágenes a color, ordenamiento vectorial y la matemática morfológica. 

\subsection{Im\'agenes RGB}

En general, una imagen es una funci\'on $f:\mathbb{Z}^2 \rightarrow \mathbb{Z}^n$.  Cada par $(u,v) \in \mathbb{Z}^2$ es un pixel, y $f(u,v) \in \mathbb Z^n $ es el color de la imagen en el pixel $(u,v)$. En particular, una imagen RGB (rojo, verde y azul, por sus siglas en inglés) con una profundidad de color de k bits es, $f(u,v) = (R,G,B)$, donde $R,G,B \in \{0,1,...,2^k-1\}$ es la intensidad del componente y $f(u,v)$ es el color resultante de mezclar estos componentes en el pixel $(u,v)$. %Si el vector $\mathbb Z^3$ es una tripleta correspondiente a los componenetes rojo, verde y azul, la imagen $f$ es llamada imagen  RGB.
La imagen $f$ puede representarse de manera digital como un arreglo $M \times N \times 3$ , donde cada pixel $(u,v)$ tiene como valor una tripleta $(R,G,B)$ \cite{gonzales2004digital}. Una imagen RGB puede ser vista como una ``pila'' de tres im\'agenes en escala de grises (ver Figura \ref{fig:ImagenRGB}) que, cuando se alimenta a las entradas de color rojo, verde y azul del monitor de color, produce una imagen de color en la pantalla \cite{gonzales2004digital}.


\begin{figure}[htbp]
	\centering
		\includegraphics[scale=0.65]{fig/ImagenRGB.png}
	\caption{RGB Image}
	\label{fig:ImagenRGB}
\end{figure}

%\subsection{Histogramas}


%\subsubsection{Histograma de color}
%El histograma a color  de una imagen $f $ en un dominio $D$  es una función discreta definida como sigue:
	
%	\begin{equation}
%	\label{eq:histograma_color}
%	H^D_{f}(C) = n_C,
%	\end{equation} donde  $C = (C_1, C_2, C_3)$, $C_i \in \{0, ..., 2^k - 1\}$ y $n_C$ es la cantidad de veces que aparece el color $C$ en el dominio $D$ de la imagen $f$. 
	
%	Etiquetemos a cada color dentro del espacio discreto con los superíndices $a$  y $b$. El color $C^a$  es menor al color $C^b$ si $C^a_1 < C^b_1$, $C^a_2 < C^b_2$ y  $C^a_3 < C^b_3$. 
	
%	El objetivo es fijar un orden de colores en el espacio de color discreto, normalmente utilizando el orden natural ascendente (Cuadro \ref{table:orden_color}).
	
%	\begin{table}[!htpb]
%		\centering
%		\caption{Orden natural de colores}
%		\label{table:orden_color}
%		\begin{tabular}{|c|c|}
%			\hline
%			$a$   & $C^a$                                                                     \\ \hline

%			0   & (0, 0, 0)                                                             \\ \hline
%			1   & (0, 0, 1)                                                             \\ \hline
%			2   & (0, 0, 2)                                                             \\ \hline
%			3   & (0, 0, 3)                                                             \\ \hline
%			\vdots & \vdots                                                                      \\ \hline
%			$2^{3k}-2$ & ($2^k-2$, $2^k-2$, $2^k-2$) \\ \hline
%			$2^{3k}-1$   & ($2^k-1$, $2^k-1$, $2^k-1$) \\ \hline
%		\end{tabular}
%	\end{table}

%	El histograma de color acumulado $\bar{H}^D_f$ de la imagen $f$ un dominio $D$ es definido en términos del histograma a color $H^D_{f}$: 
	
%	\begin{equation}
%	\bar{H}^D_f(C^b) = \sum_{C^a \leq C^b} H^D_f(C^a)
%	\end{equation}




\subsection{Filtrado  de im\'agenes}
El filtrado de im\'agenes abarca todas las t\'ecnicas dentro del procesamiento de im\'agenes, que a partir de una imagen de entrada, se obtenga otra imagen donde se elimine, se enfatice o resalte algunas caracter\'isticas de la imagen de entrada. 
Un filtro $F$ de una imagen digital a color $f$ se puede expresar como:

\begin{equation}
\label{Filtrado} 
      g(u,v) = F\{f(u,v)\}
\end{equation}
donde $f(u,v)$ es un color de la imagen de entrada, $g(u,v)$ es un color de la imagen de salida y $F$ es el filtro definido sobre una ventana del pixel $(u,v)$.


Los filtros de orden son operaciones de vecindad no lineal, donde una funci\'on es aplicada al vecindario de cada pixel. La idea es mover una ventana centrada en el pixel, ya sea un rectangulo (usualmente un rectangulo con lados impares) o otra forma sobre una imagen dada. Al hacer esto, creamos una nueva imagen cuyo p\'ixeles son el resultado de obtener un valor de los los colores bajo la m\'ascara previamente ordenados (Figura \ref{fig:Filt}). 


\begin{figure}[htbp]
	\centering
		\includegraphics[scale=0.5]{fig/Filt.jpg}
	\caption{Filtrado de la imagen digital.}
	\label{fig:Filt}
\end{figure}



Por ejemplo un pixel de la nueva imagen puede ser resultado de obtener la mediana, el m\'inimo o m\'aximo de los colores ordenados en la ventana de la imagen procesada.  
La combinaci\'on de la ventana y la funci\'on es llamada filtro. 

\subsection{Ordenamiento}
El concepto de orden juega un rol fundamental para utilizar un filtro de orden, o poder definir las operaciones b\'asicas de la matem\'atica morfol\'ogica. Para un estudio profundo de la teor\'ia de orden el lector puede ver \cite{serra1993anamorphoses}.

%Una relaci\'on binaria $\leq$ en un dominio $D$ se llama:

%\begin{enumerate}
%\item \label{reflexiva} reflexiva si $C\leq C'$, $\forall C\in D$		 
%\item \label{antisimetrica} antisim\'etrica si $C \leq C' \wedge C' \leq C$\,$\Rightarrow$ \, $C=C'$,  $\forall C,C'\in D$
%\item \label{transitiva} transitiva si $C \leq C'$ $\wedge$ $C' \leq C''$\,$\Rightarrow$ \, $C \leq C''$,  $\forall C,C',C''\in D$
%\item \label{total} total si $C' \leq C$ $\vee$ $C' \leq C$,  $\forall C,C'\in D$
%\end{enumerate}
%Una relaci\'on binaria $\leq$ es llamada de \emph{pre-orden} si cumple con \ref{reflexiva} y \ref{transitiva}; si a su vez cumple con \ref{antisimetrica} se convierte en una relaci\'on de \emph{orden}. Si adicionalmente cumple con \ref{total}, es denotada como \emph{total}, si no lo hace como \emph{parcial}.
  
%La estructura en un espacio de color est\'a dada por un ret\'iculo completo. Un ret\'iculo completo ${\cal L}$ es un conjunto no vac\'io con orden parcial  ${\cal R} $ tal que cualquier subconjunto no vac\'io  ${\cal P}$ de  ${\cal L}$ tiene un \'infimo  y tiene un supremo. 

De acuerdo con el art\'iculo \cite{barnett1976ordering} las t\'ecnicas de ordenamiento vectorial se pueden clasificar en los siguientes grupos:
\begin{itemize}
\item Ordenamiento marginal (Ordenamiento M): El ordenamiento marginal compara cada componente del color de manera independiente.
\item Ordenamiento condicional (Ordenamiento C): Los vectores son ordenados por medio de algun componente marginal, seleccionado secuencialmente de acuerdo con diferentes condiciones. El orden lexicogr\'afico es un ejemplo bastante conocido de Ordenamiento C que emplea todos los componentes disponibles de los vectores dados.
\item Ordenamiento Parcial (Ordenamiento P): Este ordenamiento est\'a basado en la partici\'on de los vectores en grupos de equivalencia, tal que entre los grupos existe un orden. En este caso, ``parcial'' es un abuso de terminolog\'ia, ya que hay ordenamientos totales que pertenecen a esta clase en particular. 
\item Ordenamiento Reducido (Ordenamiento R): Los vectores se reducen primeramente a valores escalares y luego clasificados de acuerdo a su orden escalar natural. Por ejemplo, un ordenamiento R en $\mathbb{Z}^n$ podr\'ia consistir en definir primero una transformaci\'on $T:\mathbb{Z}^n\rightarrow \mathbb{R}$ y luego ordenar los colores con respecto al orden escalar de su proyecci\'on en $\mathbb{Z}^n$ por $T$.
\end{itemize}

En la pr\'actica hay dos m\'etodos generales de procesamiento para im\'agenes a color: marginal y vectorial.

El procesamiento marginal consiste en el procesamiento por separado de cada componente de la imagen. A pesar de su sencillez, el procesamiento marginal tiene dos desventajas \cite{aptoula2007comparative}: 
\begin{itemize}
    \item La correlaci\'on entre los componentes es totalmente ignorado.
    \item Crea falsos colores despu\'es de su procesamiento.
\end{itemize}

La utilizaci\'on del procesamiento marginal es inadecuado para im\'agenes con componentes altamente correlacionados (por ejemplo, im\'agenes de color RGB) \cite{astola1990vector}. Por tal motivo este trabajo se concentrar\'a en el procesamiento vectorial que se explicar\'a a continuaci\'on. 

%\begin{figure}
%	\centering
%		\includegraphics[scale=0.5]{fig/resultado.png}
%		\caption{Imagen original y su erosión utilizando el ordenamiento marginal. En la erosión aparecen %nuevas cromaticidades.}
%	\label{fig:resultado}
%\end{figure}

El procesamiento vectorial procesa todos los componentes disponibles globalmente y de forma simult\'anea.
Dado que los vectores (forma en que se representa un color) son considerados como los nuevas unidades de procesamiento, la correlaci\'on entre los diferentes componentes ya no es ignorada. Sin embargo, en comparaci\'on con su contraparte marginal, el inconveniente m\'as importante del enfoque vectorial es principalmente la necesidad de adaptar los algoritmos existentes con el fin de acomodar a datos vectoriales \cite{aptoula2007comparative}. 


El procesamiento vectorial puede tener dos enfoques:
 
\begin{itemize}
    \item El enfoque basado en relaci\'on de pre-orden.
    \item El enfoque basado en relaci\'on de orden.
\end{itemize}

El enfoque basado en relaci\'on de pre-orden, es el conjunto de enfoques que no cumplen la propiedad antisim\'etrica.  As\'i colores distintos eventualmente pueden llegar a ser equivalentes. De manera a resolver las ambiguedades existentes, es necesario medidas adicionales. El m\'etodo principal de ordenamiento de este enfoque est\'a basado en el Ordenamiento Reducido (Ordenamiento R), donde los colores son reducidos a valores escalares correspondientes a su norma, o distancia a alg\'un color de referencia. 


El enfoque basado en relaci\'on de orden, a su vez puede ser parcial o total. Si la relaci\'on es parcial, existir\'an colores que no podr\'an ser comparados. 

La relaci\'on de orden total presenta dos principales ventajas. Primero, todos los colores son comparables, y segundo, no existen colores distintos que pueden ser equivalentes. Debido a esto, la mayor\'ia de los trabajos est\'an basados en enfoques de relaci\'on de orden total \cite{aptoula2007comparative}. En particular, el orden lexicogr\'afico (Ordenamiento C), junto con sus variantes se encuentra entre las opciones m\'as implementadas.  
 


\subsection{Matem\'atica morfol\'ogica}

Las operaciones de matemática morfológica están basadas en dos operadores básicos: erosión y dilatación. Ambos operadores son filtros que se pueden definir a partir del m\'inimo y el m\'aximo dentro de una ventana llamada elemento estructurante \cite{serra1986introduction}. A partir de la erosi\'on y la dilataci\'on se puede extender toda la matem\'atica morfol\'ogica. Los operadores morfol\'ogicos deben de cumplir ciertas propiedades de manera teórica, como ser anti-extensivo o extensivo, idempotentes, homot\'opicos y crecientes \cite{serra1986introduction}.

Dada una imagen digital $f$ y una ventana $B$, llamada elemento estructurante. La erosi\'on ($\varepsilon$) y la dilataci\'on ($\delta$) de la imagen $f$ por $B$ puede expresarse como:

\begin{equation}
\varepsilon(f,B)(u,v)  = \min_{(s,t) \in B} \{f(u-s,v-t) + B(s,t) \}
\end{equation}

\begin{equation}
\delta(f,B)(u,v)  =  \max_{(s,t) \in B} \{f(u+s,v+t) - B(s,t) \}
\end{equation}

Denotamos $\delta(f,B)$ y $\varepsilon(f,B)$ como la dilataci\'on y la erosi\'on respectivamente para todos los pixeles $(u,v)$ de la imagen $f$. La combinación de la erosión y dilatación produce otros operadores como la apertura y el cierre. La apertura suaviza las regiones brillantes de la imagen. El cierre suaviza las zonas oscuras de la imagen. La apertura $\circ$ y el cierre $\bullet$ de $f$ por $B$ son definidas basadas en dilataci\'on y erosi\'on como sigue:

\begin{equation}
f\circ B = \delta(\varepsilon(f,B),B),
\end{equation}

\begin{equation}
f\bullet B = \varepsilon(\delta(f,B),B).
\end{equation}

Basado en la apertura y el cierre, se define la transformada top-hat. La transformada top-hat clara ($WTH$) podría extraer regiones brillantes de la imagen y la transformada top-hat oscura ( $BTH$ )
podría extraer zonas oscuras. Las transformadas $WTH$ y $BTH$ son definidas para una imagen $f$ como sigue:

\begin{equation}
WTH(f) = f - f\circ B,
\end{equation}

\begin{equation}
BTH(f) = f\bullet B - f. 
\end{equation}

La extensi\'on de la matem\'atica morfol\'ogica imágenes a color es todav\'ia un problema abierto \cite{aptoula2007pseudo}, principalmente por el incoveniente de que no existe un orden natural entre los vectores, y que los colores pueden representarse de diversas formas (formando distintos espacios de color). Al no exister un orden natural entre los colores, no es sencillo definir los operadores básicos de erosión y dilatación.  

En la siguiente secci\'on se presenta una estrategia de ordenamiento de colores RGB, teniendo en cuenta m\'etricas extraidas de cada componente de color, de manera a establecer ponderaciones de los componentes a partir de informaci\'on propia de la imagen.



 \section{Ordenamiento propuesto}

A partir de la imagen RGB se define la función histograma, la cual corresponde a la distribución de frecuencia de los valores que puede tomar una imagen $f$, ya sea en un plano o en 3 dimensiones (R, G, B).  El histograma del j-ésimo componente de la imagen a color $f$ (R, G o B) es una funci\'on discreta $h_{f_j}^{D}$ definida como:% $h:\{0,1,\dots,t_{max}\}\times \{C_1,C_2,\dots,C_n\}\rightarrow \mathbb N$ definida como:

\begin{equation}
\label{histograma}
   h_{f_j}^{D}(i) = n_i,
\end{equation} 
donde ${i}$ representa el $i-esimo$ nivel de intensidad en el rango $\{0,1,...,2^k-1\}$ del componente $j$, y $n_i$ es el n\'umero de pixeles en la imagen $f$ cuyo nivel de intensidad es $i$ en el componente $j$ dentro del dominio $D$ (subconjunto de pixeles $(u,v)$ dentro de la imagen $f$).

La probabilidad de aparici\'on $p_{f_j}^{D}(i)$ de cada nivel de intensidad $i$ en el componente $j$ de la imagen $f$ dentro del dominio $D$ es definida como:
\begin{equation}
\label{probabilidad}
   p_{f_j}^{D}(i) = \frac{h_{f_j}^{D}(i)}{n},
\end{equation} Donde $n = n_0 + n_1 + ... + n_{255}$, es decir la cantidad total de pixeles de la imagen $f$ dentro del dominio $D$. 

De manera a evitar darle la mayor prioridad a un componente del vector que representa el color, se ubica un nuevo valor en la primera posici\'on de la cascada lexicogr\'afica correspondiente a una transformaci\'on obtenida a partir de m\'etricas asociadas a cada componente $(R,G,B)$.
Los colores RGB son reducidos a un valor escalar. Para tal efecto, se define primero una transformaci\'on $T:\mathbb{Z}^3 \rightarrow \mathbb{R}$ y luego se ordena los colores con respecto al orden escalar de su proyecci\'on en $\mathbb{Z}^3$ por $T$.
La reducci\'on de un color $
C=(C_1,C_2,C_3)$ se consigue por medio del producto interno del color $C$ con un vector de pesos $w=(w_1,w_2,w_3) $, es decir:

\begin{equation}
\label{Transformacion}
T(C)= \sum_{l=1}^3(w_l \times C_l)
\end{equation}  
donde $l$ es el \'indice del componente del color $C$ y $w_l \in \mathbb{R}$. 

Dos colores, $C=(C_1,C_2,C_3)$ y $C'=(C_1^{'},C_2^{'},C_3^{'})$, con $C\neq C'$, pueden tener la misma transformaci\'on, es decir $T(C) = T(C')$.  Por lo tanto, la transformaci\'on se utiliza como primer componente del orden lexicogr\'afico:

\begin{equation}
\label{Mio} 
 C\leq C'\Leftrightarrow [T(C),C_1,C_2,C_3] \leq_L [T(C'),C_1^{'},C_2^{'},C_3^{'}]
\end{equation} donde $\leq_L$ indica la relación $\leq$ según el orden lexicográfico.	

Oportunamente, despu\'es de la transformaci\'on se podr\'ia variar el orden de prioridad de los componentes de color.
Los valores del vector $w$ son obtenidos de aplicar una funci\'on $\phi \in \mathbb{R}$ sobre el histograma de cada componente en un dominio $D$ de la imagen $f$, es decir $w_1 = \phi(h_{f_1}^D)$, $w_2 = \phi(h_{f_2}^D)$, $w_3 = \phi(h_{f_3}^D)$, con $f_1$ = componente $R$, $f_2$ = componente $G$ y $f_3$ =  componente $B$.

La funci\'on $\phi$ puede ser obtenida a tr\'aves de aplicarle cualquier m\'etrica (por ejemplo estad\'istica) al histograma de cada componente  $(R,G,B)$, de manera de darle mayor peso a aquel componente cuya m\'etrica tenga mayor valor en un dominio $D$ espec\'ifico (puede ser toda la imagen o parte de la misma). 

 \section{Resultados experimentales}

En esta secci\'on se llevar\'a a cabo una serie de pruebas comparativas, con el fin de medir los rendimientos relativos de diferentes m\'etodos de ordenamiento del estado del arte junto al  ordenamiento propuesto, en tres aplicaciones de procesamiento de im\'agenes. Las aplicaciones seleccionadas fueron la eliminaci\'on de ruido, estiramiento de contraste y caracterización texturas para su posterior clasificación.
M\'as precisamente los m\'etodos de ordenamiento que participaron de las diferentes pruebas fueron:
el ordenamiento lexicogr\'afico cl\'asico, el ordenamiento $\alpha$-lexicogr\'afico \cite{zamora2001comparative}, el ordenamiento $\alpha$-modulo lexicogr\'afico \cite{angulo2003morphological}, ordenamiento lexicogr\'afico I→H→S, $(Href=0º)$ \cite{ortiz2004gaussian}, la distancia euclidiana al color $(0,0,0)$ en el espacio de color L*a*b* y RGB \cite{ortiz2002procesamiento}, y el entrelazado de bits \cite{chanussot1997bit}. 
Todas las im\'agenes utilizadas en las diferentes pruebas fueron de $8$ bits. 

La funci\'on $\phi$ aplicada al histograma de cada componente $j$ de la imagen $f$ en todas las pruebas son:

\begin{itemize}
    \item Promedio ($Me$): Es la sumatoria de todos los niveles de intensidades $i$ que aparecen en el dominio $D$ sobre la cantidad total $n$ de pixeles que se encuentran en $D$:
		
\begin{equation}
\label{promedio}
   Me(h_{f_j}^D) = \sum_{i=0}^{255}\frac{i\times h_{f_j}^D(i)}{n},
\end{equation}		
Donde $n = n_0 + n_1 + ... + n_{255}$.

    \item M\'inimo($Min$): es el menor nivel de intensidad $i$ en el dominio $D$:
\begin{equation}
\label{minimo}
	Min(h_{f_j}^D) = \min\{i|h_{f_j}^D(i)>0\}
\end{equation}		
		\item M\'aximo($Max$): es el mayor nivel de intensidad $i$ en el dominio $D$:
\begin{equation}
\label{maximo}
   Max(h_{f_j}^D) = \max\{i|h_{f_j}^D(i)>0\}
\end{equation}		
 		\item Moda Minimo ($minM_o$): es el menor nivel de intensidad $i$ que aparece m\'as veces en el dominio $D$, es decir el menor nivel de intensidad $i$ que tiene mayor $p_{f_j}^{D}(i)$:
\begin{equation}
\label{ModaMinimo}
   minM_o(h_{f_j}^D) = \min\{i|h_{f_j}^D(i) \geq h_{f_j}^D(i'), \forall i\neq i'\}
\end{equation}		
\item Moda M\'aximo ($maxM_o$): es el mayor nivel de intensidad $i$ que aparece m\'as veces en el dominio $D$, es decir el mayor nivel de intensidad $i$ que tiene mayor $p_{f_j}^{D}(i)$:
\begin{equation}
\label{ModaMinimo}
   maxM_o(h_{f_j}^D) = \max\{i|h_{f_j}^D(i) \geq h_{f_j}^D(i'), \forall i\neq i'\}
\end{equation}				
    
		
		\item Varianza ($Var$): es la varianza de los niveles de intensidad $i$ en el dominio $D$:
  \begin{equation}
\label{varianza}
   Var(h_{f_j}^D) = \sum_{i=0}^{255}\frac{h_{f_j}^D(i)\times(i - Me(h_{f_j}^D))^2}{n}
\end{equation}		
		\item Suavidad($R$): Medida de suavidad relativa de la intensidad en el dominio $D$:
 \begin{equation}
\label{Suavidad}
   R(h_{f_j}^D) = 1-\frac{1}{1+Var(h_{f_j}^D)}
\end{equation}		
		

\end{itemize}


Un par\'ametro a ser definido es el dominio a ser tenido en cuenta para el c\'alculo
de los pesos $w_l$. Las distribuciones de dominio que fueron empleadas para las diferentes aplicaciones son explicadas a continuaci\'on.

\subsection{Vecindario como Dominio}


El dominio $D$ donde se aplica la funci\'on $\phi$ aplicada al histograma de cada componente $h_{f_j}^D$ es la propia ventana $B$ (llamado elemento estructurante para matem\'atica morfol\'ologica) donde se hace la operaci\'on del filtro no lineal. En la Figura \ref{fig:Ventana B} se puede observar un dominio $D$ correspondiente a un vecindario $B$ de tama\~no 3 $\times$ 3 centrado en el pixel $(u,v)$.

\begin{figure}
	\centering
		\includegraphics[width=0.3\textwidth]{fig/VentanaB.jpg}
	\caption{Vecindario $B$ de tama\~no 3 $\times$ 3 centrado en el pixel $(u,v)$ }
	\label{fig:Ventana B}
\end{figure}

\subsection{Divisi\'on de la imagen en sub-regiones}

La imagen $f$ se divide en sub-regiones $W_1,W_2,\dots W_{x}$, de manera a obtener informaci\'on local de  la imagen. 

Sea $B$ una ventana o  elemento estructurante, el dominio $D$ correspondiente a la ventana $B$ centrada en $(u,v)$ es el conjunto de sub-regiones $W_{\{1,2,...,x\}}$, que tocan alg\'un pixel de $B$.
%\begin{equation}
%\label{ventana}
%  \forall i, W_i \subseteq D_{(u,v)}\rightarrow \exists  \in D_{(u,v)}:x \in W_i
%\end{equation}   

En la Figura \ref{fig:Sub-region} la imagen se divide en 4 sub-regiones: $W_1$, $W_2$, $W_3$, $W_4$.  La regi\'on delimitada por la ventana $B$ se encuentra sombreada. Como puede verse, el dominio
$D$ sobre el cual se calcular\'an los pesos $w_l$ para $B$ ser\'a la zona correspondiente a la
sub-region $W_1$.

%\begin{figure}
%	\centering
%		\includegraphics{fig/ventanas.png}
%	\caption{Imagen divida en 4 sub-regiones}
%	\label{fig:ventanas}
%\end{figure}


Cabe destacar que la ventana del filtro no tiene por qu\'e ser de igual
tama\~no que las sub-regiones, como en este caso. En la Figura \ref{fig:Sub-regiones} se puede observar como el dominio $D$ de donde se calcular\'an los pesos pertenece a la uni\'on de las sub-regiones $W_1$ y $W_2$, ya que la ventana $B$ toca ambas. Esto se realiza para evitar que en el momento de la comparaci\'on dos colores  iguales puedan tener distintos valores, afectados por los pesos que vengan de dos sub-regiones distintas.  


\begin{figure}
	\centering
		\includegraphics[width=0.3\textwidth]{fig/Sub-region.jpg}
	\caption{Dominio cuando la ventana toca una sub-regi\'on}
	\label{fig:Sub-region}
\end{figure}




\begin{figure}
	\centering
		\includegraphics[width=0.3\textwidth]{fig/Sub-regiones.jpg}
	\caption{Dominio cuando la ventana toca m\'as de una sub-regi\'on.}
	\label{fig:Sub-regiones}
\end{figure} En el caso que el usuario solo seleccione tener una sub-regi\'on, es decir no dividir la imagen $f$, los pesos ser\'an calculados teniendo en cuenta toda la imagen $f$ como dominio $D$. 

En nuestros pruebas, las im\'agenes de entradas de tama\~no $M \times N$ pixeles, son divididas en sub-regiones $W_{\{1,2,...,x\}}$ de $\left\lfloor\frac{M}{M'}\right\rfloor$ filas y $\left\lfloor\frac{N}{N'}\right\rfloor$ columnas, donde $\left\lfloor.\right\rfloor$ denota la funci\'on piso. De esta forma, tenemos una nueva matriz de $M'$ filas y $N'$ columnas, cuyo elemento es una sub-region $W_l$.



\subsection{Aplicaci\'on 1: Eliminaci\'on de ruido}
Ruido es un t\'ermino utilizado para denominar a las modificaciones indeseadas que puede sufrir una se\~nal de cualquier naturaleza durante su captura, almacenamiento, transmisi\'on, procesamiento o conversi\'on \cite{tuzlukov2002signal}.

El ruido en im\'agenes es un producto indeseable que agrega informaci\'on err\'onea y ajena a las mismas. El ruido se presenta en las im\'agenes digitales en forma de variaciones aleatorias en el brillo o informaci\'on de color. 
Varios modelos matem\'aticos han sido desarrollados de modo a simular la generaci\'on de los distintos 
tipos de ruido existentes. 
\subsubsection{Ruidos Utilizados} 

Dadas una imagen de entrada $f$, la imagen $f'$ resultante de contaminar a $f$ con cierto tipo de ruido y un vector $z = (z_1, z_2, z_3)$ en el que cada elemento $z_l$  corresponde a una variable aleatoria; se define a los principales modelos de ruido como sigue:

\begin{itemize}
	
\item{Ruido gaussiano:}
es un ruido estadístico aditivo con función de densidad de probabilidad gaussiana \cite{davenport1958random}. El ruido gaussiano se expresa de la siguiente forma:
\begin{equation}
\label{eq:gaussian_noise}
f'(x, y) = f(x, y) + z
\end{equation}
donde cada componente $z_l$ es una variable aleatoria con distribución normal, promedio $\mu$, varianza $\sigma^2$ y representa al valor de ruido agregado.

\item{Ruido speckle:}
es un ruido multiplicativo con función de densidad de probabilidad uniforme, definido como sigue: 

\begin{equation}
\label{eq:speckle}
f'(x, y) = f(x, y) + z \ast f(x, y)  
\end{equation}
donde el operador $\ast$ simboliza el producto Hadamard o elemento a elemento. Cada elemento $z_l$ es una variable aleatoria uniformemente distribuida con promedio $\mu$ y varianza $\sigma^2$.

\item{Ruido sal y pimienta:}
este ruido, a diferencia de los ruidos gaussiano y speckle, no es aditivo ni multiplicativo con respecto a los valores de la imagen original. En las imágenes afectadas con ruido sal y pimienta los valores originales son reemplazados por valores brillantes (sal) o valores oscuros (pimienta), que corresponden a impulsos dentro de la señal. 

Los píxeles sal poseen el mínimo valor posible (cero) y los valores de los píxeles pimienta el máximo valor posible ($2^k - 1$, donde $k$ es la cantidad de bits utilizados para representar la intensidad de cada componente de color). El ruido sal y pimienta no afecta a todos los píxeles dentro de una imagen, como sucede con los ruidos gaussiano y speckle. La cantidad de píxeles de una imagen que son afectados por el ruido sal y pimienta depende del parámetro de probabilidad de ruido $p$, el cual se encuentra en el intervalo $[0, 1]$.

El ruido sal y pimienta se modela de la siguiente forma:
\begin{equation}
\label{eq:salt_and_pepper}
f'(x, y)=\left\{ \begin{array}{cl}
s & \text{, con probabilidad } p/2 \\
r & \text{, con probabilidad } p/2 \\
f(x, y) & \text{, con probabilidad } 1 - p \\
\end{array}\right.
\end{equation}
donde:\\ $s = (0, 0, 0)$ representa al ruido sal,\\ $r = (2^k - 1, 2^k - 1, 2^k - 1 )$ representa al ruido pimienta.


\end{itemize}



En la Figura \ref{fig:ruido}(a) se puede observar una imagen que es contaminada con ruido gaussiano ( \ref{fig:ruido}(b)), ruido speckle (\ref{fig:ruido}(c)), con ruido sal y pimienta (\ref{fig:ruido}(d)). 

%\begin{figure}[htbp]
%\centering
%\subfigure[Imagen original]{\includegraphics[width=73mm]{fig/Original_SinRuido.jpg}}
%\subfigure[Imagen con ruido gaussiano ($\sigma^2 = 0.05, \mu=0$)]{\includegraphics[width=73mm]{fig/img_gaussian_noise.jpg}}
%\subfigure[Imagen con ruido speckle ($\sigma^2 = 0.05, \mu=0$)]{\includegraphics[width=73mm]{fig/img_speckle_noise.jpg}}
%\subfigure[Imagen con ruido sal y pimienta ($p = 0.05$)]{\includegraphics[width=73mm]{fig/img_salt_and_pepper_noise.jpg}}
%\caption{Imagen con diferentes tipos de ruidos.} \label{fig:ruido}
%\end{figure}


\begin{figure*}
	\makebox[\linewidth][c]{%
		\begin{subfigure}[b]{0.4\textwidth}
			\centering
			\includegraphics[width=0.95\textwidth]{fig/Original_SinRuido.jpg}
			\caption{Imagen original}
			\label{fig:original_sin_ruido}
		\end{subfigure}
		\begin{subfigure}[b]{0.4\textwidth}
			\centering
			\includegraphics[width=0.95\textwidth]{fig/img_gaussian_noise.jpg}
			\caption{Imagen con ruido gaussiano ($\mu = 0; \sigma^2 = 0.05$)}
			\label{fig:con_ruido_gaussiano}
		\end{subfigure}
	}\\
	\makebox[\linewidth][c]{%
		\begin{subfigure}[b]{0.4\textwidth}
			\centering
			\includegraphics[width=0.95\textwidth]{fig/img_speckle_noise.jpg}
			\caption{Imagen con ruido speckle ($\mu = 0; \sigma^2 = 0.05$)}
			\label{fig:con_ruido_speckle}
		\end{subfigure}
			\begin{subfigure}[b]{0.4\textwidth}
				\centering
				\includegraphics[width=0.95\textwidth]{fig/img_salt_and_pepper_noise.jpg}
				\caption{Imagen con ruido sal y pimienta ($p = 0.05$)}
				\label{fig:con_ruido_sal_y_pimienta}
			\end{subfigure}
	}\\
	\caption{Imagen con diferentes tipos de ruidos.} \label{fig:ruido}
\end{figure*}





%Los par\'ametros utilizados en la generaci\'on de cada ruido fueron:

%\begin{itemize}
%	\item Ruido gaussiano: La media  $\mu=0$ y  la varianza $\sigma^2$ fue variando  entre los valores 0,01 y 0,17, haciendo pasos de 0,01. 
%	\item Ruido speckle: La media  $\mu=0$ y  la varianza $\sigma^2$ fue variando  entre los valores 0,01 y 0,17, haciendo pasos de 0,01. 
%	\item Ruido sal y pimienta: El valor de probabilidad $p$  de ocurrencia de un ruido sal o pimienta fue variando  entre los valores 0,01 y 0,17, haciendo pasos de 0,01. 
%\end{itemize}

%El valor de la varianza $\sigma^2$ para los ruidos gaussiano y speckle, as\'i como el valor de probabilidad $p$ de ocurriencia de un ruido sal o pimienta fue variando  de manera a observar como se comporta el filtro a medida que se va aumentando la cantidad de ruido.

De manera a evaluar el filtro con los diferentes tipos de ordenamiento, se propone utilizar una m\'etrica utilizada en estad\'istica para medir que tan cerca est\'an los pron\'osticos o predicciones de los resultados reales \cite{willmott2005advantages}.

Dadas una imagen $f$ y su imagen filtrada $g$ de dimensiones $M \times N$ , el error absoluto promedio de la imagen filtrada est\'a dado por:
\begin{equation}
\label{MAE}
MAE(f,g) = \frac{1}{3\times M\times N}\sum_{j=1}^3 d_j
\end{equation} donde: 


\begin{equation}
d_j = \sum_{\substack{u\in \{1, ..., M\}\\ v \in \{1, ..., N\}}} |[f(u,v)]_{j} - [g(u,v)]_{j}| 
\end{equation}

%\textsl{CDS (Color Distribution Similarity, Similaridad de Distribuciones de Color):} es una medida de similaridad para imágenes digitales a color basada en la diferencia de histogramas acumulados \cite{stricker1995similarity}.  La principal mejora de la diferencia de histogramas a color acumulado con respecto a la diferencia de histogramas a color radica; en que la diferencia de histograma a color acumulado obtiene todas las imágenes con similaridad perceptual a sus histogramas a color y de esta manera disminuye la probabilidad de obtener falsos negativos.

%Para determinar la similaridad de dos imágenes $f$ y $g$, con histogramas acumulados a color $\bar{H}^D_f$ y $\bar{H}^D_g$ en un dominio $D$, se calcula la diferencia entre $\bar{H}^D_f$ y $\bar{H}^D_g$: 
%\begin{equation}
%CDS(f,g) = \sqrt{\sum_{\substack{u\in \{1, ..., M\}\\ v \in \{1, ..., N\}}} (\bar{H}^D_f(f(u, v)) - \bar{H}^D_g(g(u, v)))^2}
%\end{equation}

%\include{anexos/graficos}
\subsubsection{Resultados}

A continuaci\'on se listan los c\'odigos utilizados para abreviar los nombres de los métodos de ordenamiento que fueron objeto de experimentación:


\begin{itemize}
	\item ED: Se utiliza la Distancia Euclidiana  RGB como m\'etodo de ordenamiento de los colores \cite{ortiz2002procesamiento}.
	\item BM: Se utiliza el Entrelazado de Bits RGB como m\'etodo de ordenamiento de los colores \cite{chanussot1997bit}.
	\item LEX: Se utiliza el ordenamiento lexicogr\'afico RGB para ordenar los colores .
	\item ALEX: Se utiliza el ordenamiento $\alpha$-lexicogr\'afico RGB \cite{zamora2001comparative} para ordenar los colores.
	\item AMLEX: Se utiliza el  $\alpha$-modulo lexicogr\'afico RGB \cite{angulo2003morphological} para ordenar los colores.
	\item HLEX, se utiliza el ordenamiento Lexicográfico I→H→S para ordenar los colores.
	\item DLAB, se utiliza la Distancia en L*a*b* como m\'etodo de ordenamiento \cite{ortiz2002procesamiento}.
	\item MIN: se utiliza la ecuaci\'on \ref{Transformacion} para agregar esta transformaci\'on como primer componente de la cascada lexicogr\'afica RGB, donde:\\ 
$w = (Min(h_{f_1}^D), Min(h_{f_2}^D),Min(h_{f_3}^D))$.
	\item MAX: se utiliza la ecuaci\'on \ref{Transformacion} para agregar esta transformaci\'on como primer componente de la cascada lexicogr\'afica RGB, donde:\\ 
$w = (Max(h_{f_1}^D), Max(h_{f_2}^D),Max(h_{f_3}^D))$.
	\item MO1: se utiliza la ecuaci\'on \ref{Transformacion} para agregar esta transformaci\'on como primer componente de la cascada lexicogr\'afica RGB, donde:\\ 
$w = (minM_o(h_{f_1}^D), minM_o(h_{f_2}^D), minM_o(h_{f_3}^D))$.
	\item MO2: se utiliza la ecuaci\'on \ref{Transformacion} para agregar esta transformaci\'on como primer componente de la cascada lexicogr\'afica RGB, donde:\\ 
$w = (maxM_o(h_{f_1}^D), maxM_o(h_{f_2}^D), maxM_o(h_{f_3}^D))$.
	\item SMO: se utiliza la ecuaci\'on \ref{Transformacion} para agregar esta transformaci\'on como primer componente de la cascada lexicogr\'afica RGB, donde:\\ 
$w = (R(h_{f_1}^D), R(h_{f_2}^D), R(h_{f_3}^D))$.
	\item MEAN: se utiliza la ecuaci\'on \ref{Transformacion} para agregar esta transformaci\'on como primer componente de la cascada lexicogr\'afica RGB, donde:\\ 
$w = (Me(h_{f_1}^D), Me(h_{f_2}^D), Me(h_{f_3}^D))$.
	\item VAR: se utiliza la ecuaci\'on \ref{Transformacion} para agregar esta transformaci\'on como primer componente de la cascada lexicogr\'afica RGB, donde:\\ 
$w = (Var(h_{f_1}^D), Var(h_{f_2}^D), Var(h_{f_3}^D))$.
\end{itemize}

El filtro utilizado para eliminar los diferentes tipos de ruido fue la mediana. Este filtro consiste en ordenar los colores dentro de la ventana del filtro, y seleccionar el valor del medio para remplazar en la imagen de salida. De manera a evitar tener un falso color, el tama\~no de la ventana del filtro suele ser impar (en nuestras pruebas de $3\times 3$). Las pruebas fueron realizadas con 100 im\'agenes diferentes (im\'agenes de prueba de \cite{arbelaez2007berkeley}), contaminándolas con los ruidos: gaussiano, speckle, sal y pimienta. En el caso de los ruidos gaussiano y speckle se mantuvo el parámetro $\mu = 0$ y se varió el parámetro $\sigma^2$ entre $0.005$ y $0.165$, con incrementos de $0.02$. En el caso del ruido sal y pimienta, el parámetro de probabibilidad $p$ se varió con los mismos valores del parámetro $\sigma^2$ de los ruidos gaussiano y speckle.

La Figura \ref{fig:imagen_ejemplo_ruido_gaussiano} corresponde a la imagen original de la Figura \ref{fig:ruido} con $\sigma^2 = 0.105$ y $\mu = 0$. El resultado de aplicar los filtros de orden propuestos y el resto de los filtros evaluados sobre la imagen contaminada puede verse en la Figura \ref{fig:imagenes_resultado}.


\begin{figure}
	\centering
	\includegraphics[width=0.38\textwidth]{fig/img_ruido_gaussian_83_0_105}
	\caption{Ruido gaussiano ($\mu = 0; \sigma^2 = 0.105$)}
	\label{fig:imagen_ejemplo_ruido_gaussiano}
\end{figure}


Cabe resaltar que las imágenes filtradas con el orden propuesto y con los diferentes pesos son mejores visualmente a los del estado del arte, pero son perceptualmente muy parecidas entre si, la diferencia entre ellas se evidencia en los resultados numéricos expuestos más adelante en esta sección.

\begin{figure*}
	\makebox[\linewidth][c]{%
		\begin{subfigure}[b]{0.25\textwidth}
			\centering
			\includegraphics[width=0.95\linewidth]{fig/ALEX}
			\caption{ALEX}
			\label{fig:alex}
		\end{subfigure}
		\begin{subfigure}[b]{0.25\textwidth}
			\centering
			\includegraphics[width=0.95\linewidth]{fig/AMLEX}
			\caption{AMLEX}
			\label{fig:amlex}
		\end{subfigure}
		\begin{subfigure}[b]{0.25\textwidth}
			\centering
			\includegraphics[width=0.95\linewidth]{fig/BM}
			\caption{BM}
			\label{fig:bm}
		\end{subfigure}
		\begin{subfigure}[b]{0.25\textwidth}
			\centering
			\includegraphics[width=0.95\linewidth]{fig/DLAB}
			\caption{DLAB}
			\label{fig:dlab}
		\end{subfigure}
	}\\
	\makebox[\linewidth][c]{%
		\begin{subfigure}[b]{0.25\textwidth}
			\centering
			\includegraphics[width=0.95\linewidth]{fig/ED}
			\caption{ED}
			\label{fig:ed}
		\end{subfigure}
		\begin{subfigure}[b]{0.25\textwidth}
			\centering
			\includegraphics[width=0.95\linewidth]{fig/HLEX}
			\caption{HLEX}
			\label{fig:hlex}
			\phantomcaption
		\end{subfigure}
		\begin{subfigure}[b]{0.25\textwidth}
			\centering
			\includegraphics[width=0.95\linewidth]{fig/LEX}
			\caption{LEX}
			\label{fig:lex}
			\phantomcaption
		\end{subfigure}
		\begin{subfigure}[b]{0.25\textwidth}
			\centering
			\includegraphics[width=0.95\linewidth]{fig/MAXM5}
			\caption{MAX}
			\label{fig:max}
			\phantomcaption
		\end{subfigure}
	}\\
	\makebox[\linewidth][c]{%
		\begin{subfigure}[b]{0.25\textwidth}
			\centering
			\includegraphics[width=0.95\linewidth]{fig/MEANM5}
			\caption{MEAN}
			\label{fig:mean}
			\phantomcaption
		\end{subfigure}
		\begin{subfigure}[b]{0.25\textwidth}
			\centering
			\includegraphics[width=0.95\linewidth]{fig/MINM5}
			\caption{MIN}
			\label{fig:min}
			\phantomcaption
		\end{subfigure}
		\begin{subfigure}[b]{0.25\textwidth}
			\centering
			\includegraphics[width=0.95\linewidth]{fig/MOD2M5}
			\caption{MO2}
			\label{fig:mo2}
			\phantomcaption
		\end{subfigure}
		\begin{subfigure}[b]{0.25\textwidth}
			\centering
			\includegraphics[width=0.95\linewidth]{fig/MODM5}
			\caption{MO1}
			\label{fig:mo1}
			\phantomcaption
		\end{subfigure}
	}\\
	\makebox[\linewidth][c]{%
		\begin{subfigure}[b]{0.25\textwidth}
			\centering
			\includegraphics[width=0.95\linewidth]{fig/SMOM5}
			\caption{SMO}
			\label{fig:smo}
			\phantomcaption
		\end{subfigure}
		\begin{subfigure}[b]{0.25\textwidth}
			\centering
			\includegraphics[width=0.95\linewidth]{fig/VARM5}
			\caption{VAR}
			\label{fig:var}
			\phantomcaption
		\end{subfigure}
	}\\
	\caption{Resultados de aplicar los distintos filtros evaluados sobre la imagen de la Figura \ref{fig:imagen_ejemplo_ruido_gaussiano}. La imagen fue dividida en sub-regiones de 5 x 5.}	
	\label{fig:imagenes_resultado}
\end{figure*}

Por cada ruido y métrica se presenta una tabla de resultados y un gráfico de curvas de tendencia de cada filtro con respecto a la variación del parámetro de ruido ($\sigma^2$ para ruido gaussiano y speckle, y $p$ para ruido sal y pimienta). Cada punto representa al promedio de métrica obtenido por ese filtro para un cierto valor de parámetro de ruido $(\sigma^2,\,MAE)$ o $(p,\,MAE)$. La curva correspondiente a un filtro se obtiene uniendo a cada par de puntos sucesivos de dicho filtro con la línea (recta) que pasa por ambos puntos. Esto se hace de forma  a poder visualizar la tendencia como una función continua. Las tablas de resultados ordenan a los filtros según la suma total de todos los puntos en el gráfico de curvas. Los filtros que aparecen en los primeros lugares de las tablas son aquellos que poseen los menores valores de la sumatoria total de MAE obtenida por cada filtro. En algunos casos, los filtros del estado del arte se comportan mejor para menores valores de parámetro de ruido pero se ven superados por los filtros propuestos para mayores valores de parámetro de ruido. 

 Los resultados de esta sección diferencian a los filtros de orden de cada peso de acuerdo a su configuración de dominio. Como referencia, a los códigos de los filtros propuestos se les agrega el sufijo "WX", donde X es un número que representa la cantidad de sub-regiones en que fue dividida la imagen, con M'= $\sqrt{X}$ y N'= $\sqrt{X}$. Cuando el vecindario (marcado por la ventana del filtro) es el dominio, se utiliza el sufijo "B".

%Para poder seleccionar varias columnas de una vez
\pgfplotstableset{
	my multistyler/.style 2 args={
		@my multistyler/.style={display columns/##1/.append style={#2}},
		@my multistyler/.list={#1}
	}
}

\pgfplotstableread[
col sep = tab,
header=has colnames,
]
{anexos/files/color_and_marker_by_filter.txt}{\colorbyfilter}



%mae gaussiano
\pgfplotstableread[
col sep = tab,
header=has colnames,
comment chars={P},
]
{anexos/files/ventanas_mae_gaussian.txt}{\ventanasmaegaussian}
\pgfplotstablegetcolsof{\ventanasmaegaussian}
\pgfmathsetmacro{\C}{\pgfplotsretval - 1}
\pgfmathsetmacro{\B}{\C-1}
\pgfplotstablegetrowsof{\ventanasmaegaussian}
\pgfmathsetmacro{\R}{\pgfplotsretval - 1}
\pgfmathsetmacro{\F}{6} %cantidad de filas a mostrar en las tablas de resumen

\pgfplotstabletranspose[
colnames from=p,
input colnames to=p
]\ventanasmaegaussiannew{anexos/files/ventanas_mae_gaussian.txt}


\begin{table}[!h]
	\centering
	\caption{Ruido gaussiano. Sumatoria de MAE por $\sigma^2$}
	\pgfkeys{/pgf/number format/precision=4}
	\pgfplotstabletypeset[
	col sep=tab,
	%header=has colnames,
	columns = {[index]0, [index]10},
	columns/P/.style = {column name = Suma},
	columns/p/.style = {column name = Filtro},
	%skip rows between index={6}{25},
	every first column/.style={
		string type,
		column type/.add={|}{|},
	},
	every head row/.style={
		before row=\hline,
		after row=\hline
	},
	every last column/.style={column type/.add={}{|}},
	every last row/.style={
		after row=\hline,
	},
	]{\ventanasmaegaussiannew}
	\label{tabla:gaussian_mae}
\end{table}

%mae sal y pimienta
\pgfplotstableread[
col sep = tab,
header=has colnames,
comment chars={P},
]
{anexos/files/ventanas_mae_salt_and_pepper.txt}{\ventanasmaesaltandpepper}

\pgfplotstabletranspose[
colnames from=p,
input colnames to=p
]\ventanasmaesaltandpeppernew{anexos/files/ventanas_mae_salt_and_pepper.txt}


\begin{table}[htbp]
	\centering
	\caption{Ruido sal y pimienta. Sumatoria de MAE por $p$}
	\pgfkeys{/pgf/number format/precision=4}
	\pgfplotstabletypeset[
	col sep=tab,
	%header=has colnames,
	columns = {[index]0, [index]10},
	columns/P/.style = {column name = Suma},
	columns/p/.style = {column name = Filtro},
	%skip rows between index={6}{25},
	every first column/.style={
		string type,
		column type/.add={|}{|},
	},
	every head row/.style={
		before row=\hline,
		after row=\hline
	},
	every last column/.style={column type/.add={}{|}},
	every last row/.style={
		after row=\hline,
	},
	]{\ventanasmaesaltandpeppernew}
	\label{tabla:salt_and_pepper_mae}
\end{table}

%mae speckle
\pgfplotstableread[
col sep = tab,
header=has colnames,
comment chars={P},
]
{anexos/files/ventanas_mae_speckle.txt}{\ventanasmaespeckle}

\pgfplotstabletranspose[
colnames from=p,
input colnames to=p
]\ventanasmaespecklennew{anexos/files/ventanas_mae_speckle.txt}


\begin{table}[htbp]
	\centering
	\caption{Ruido speckle. Sumatoria de MAE por $\sigma^2$}
	\pgfkeys{/pgf/number format/precision=4}
	\pgfplotstabletypeset[
	col sep=tab,
	%header=has colnames,
	columns = {[index]0, [index]10},
	columns/P/.style = {column name = Suma},
	columns/p/.style = {column name = Filtro},
	%skip rows between index={6}{25},
	every first column/.style={
		string type,
		column type/.add={|}{|},
	},
	every head row/.style={
		before row=\hline,
		after row=\hline
	},
	every last column/.style={column type/.add={}{|}},
	every last row/.style={
		after row=\hline,
	},
	]{\ventanasmaegaussiannew}
	\label{tabla:speckle_mae}
\end{table}


\begin{figure*}[htbp]
	\centering
	\begin{tikzpicture}[spy using outlines={circle, magnification=9.5, size=3.5cm, connect spies}]
	\begin{axis}[
	xmax = 0.175,
	legend style={mark options={scale=1.5}},
	xlabel={$\sigma^2$},
	ylabel={MAE},
	legend pos=outer north east,
	height=0.5\textwidth,
	width=0.6\textwidth,
	tick label style={/pgf/number format/fixed,
		/pgf/number format/precision=4}
	]
	\foreach \n in {\C,\B,...,1} {
		\pgfplotstablegetcolumnnamebyindex{\n}\of{\ventanasmaegaussian}\to{\colname}
		\pgfplotstablegetelem{0}{\colname}\of{\colorbyfilter}
		\let\color=\pgfplotsretval
		\pgfplotstablegetelem{1}{\colname}\of{\colorbyfilter}
		\let\marca=\pgfplotsretval
		\edef\temp{
			\noexpand\addplot+[\color, mark=\marca, solid, mark options={mark size = 1.5, fill=\color!70}] 
		}
		\temp	
		table[x=p, y index=\n]{\ventanasmaegaussian};
		
		\addlegendentryexpanded{\colname}
	}%
	\pgfplotstablegetelem{\R}{[index]1}\of{\ventanasmaegaussian}
	\let\ytozoom=\pgfplotsretval
	\coordinate (spypoint) at (axis cs:0.165,\ytozoom);
	\coordinate (spyviewer) at (axis cs:0.12,17);
	\spy on (spypoint) in node [fill=white] at (spyviewer);
	\end{axis}
	\end{tikzpicture}
	\caption{Ruido gaussiano. MAE por $\sigma^2$}
	\label{fig:gaussian_mae}
\end{figure*}

\begin{figure*}[htbp]
	\centering
	\begin{tikzpicture}[spy using outlines={circle, magnification=9.5, size=3.5cm, connect spies}]
	\begin{axis}[
	xmax = 0.175,
	legend style={mark options={scale=1.5}},
	xlabel={$p$},
	ylabel={MAE},
	legend pos=outer north east,
	height=0.5\textwidth,
	width=0.6\textwidth,
	tick label style={/pgf/number format/fixed,
		/pgf/number format/precision=4}
	]
	\foreach \n in {\C,\B,...,1} {
		\pgfplotstablegetcolumnnamebyindex{\n}\of{\ventanasmaesaltandpepper}\to{\colname}
		\pgfplotstablegetelem{0}{\colname}\of{\colorbyfilter}
		\let\color=\pgfplotsretval
		\pgfplotstablegetelem{1}{\colname}\of{\colorbyfilter}
		\let\marca=\pgfplotsretval
		\edef\temp{
			\noexpand\addplot+[\color, mark=\marca, solid, mark options={mark size = 1.5, fill=\color!70}] 
		}
		\temp	
		table[x=p, y index=\n]{\ventanasmaesaltandpepper};
		
		\addlegendentryexpanded{\colname}	
	}%
	\pgfplotstablegetelem{\R}{[index]1}\of{\ventanasmaesaltandpepper}
	\let\ytozoom=\pgfplotsretval
	\coordinate (spypoint) at (axis cs:0.165,\ytozoom);
	\coordinate (spyviewer) at (axis cs:0.04,6.15);
	\spy on (spypoint) in node [fill=white] at (spyviewer);
	\end{axis}
	\end{tikzpicture}
	\caption{Ruido sal y pimienta. MAE por $p$}
	\label{fig:salt_and_pepper_mae}
\end{figure*}


\begin{figure*}[htbp]
	\centering
	\begin{tikzpicture}[spy using outlines={circle, magnification=9.5, size=3.5cm, connect spies}]
	\begin{axis}[
	xmax = 0.175,
	legend style={mark options={scale=1.5}},
	xlabel={$p$},
	ylabel={MAE},
	legend pos=outer north east,
	height=0.5\textwidth,
	width=0.6\textwidth,
	tick label style={/pgf/number format/fixed,
		/pgf/number format/precision=4}
	]
	\foreach \n in {\C,\B,...,1} {
		\pgfplotstablegetcolumnnamebyindex{\n}\of{\ventanasmaespeckle}\to{\colname}
		\pgfplotstablegetelem{0}{\colname}\of{\colorbyfilter}
		\let\color=\pgfplotsretval
		\pgfplotstablegetelem{1}{\colname}\of{\colorbyfilter}
		\let\marca=\pgfplotsretval
		\edef\temp{
			\noexpand\addplot+[\color, mark=\marca, solid, mark options={mark size = 1.5, fill=\color!70}] 
		}
		\temp	
		table[x=p, y index=\n]{\ventanasmaespeckle};
		
		\addlegendentryexpanded{\colname}	
	}%
	\pgfplotstablegetelem{\R}{[index] 1}\of{\ventanasmaespeckle}
	\let\ytozoom=\pgfplotsretval
	\coordinate (spypoint) at (axis cs:0.165,\ytozoom);
	\coordinate (spyviewer) at (axis cs:0.12,13);
	\spy on (spypoint) in node [fill=white] at (spyviewer);
	\end{axis}
	\end{tikzpicture}
	\caption{Ruido speckle. MAE por $\sigma^2$}
	\label{fig:speckle_mae}
\end{figure*}

%
%%cds gaussiano
%\pgfplotstableread[
%col sep = tab,
%header=has colnames,
%comment chars={P},
%]
%{anexos/files/ventanas_cds_gaussian.txt}{\ventanascdsgaussian}
%\pgfplotstablegetcolsof{\ventanascdsgaussian}
%\pgfmathsetmacro{\C}{\pgfplotsretval-1}
%
%
%\begin{figure*}
%	\centering
%	\begin{tikzpicture}[spy using outlines={circle, magnification=9.5, size=3.5cm, connect spies}]
%	\begin{axis}[
%	xmax = 0.175,
%	legend style={mark options={scale=1.5}},
%	xlabel={$\sigma^2$},
%	ylabel={CDS},
%	legend pos=outer north east,
%	height=0.5\textwidth,
%	width=0.6\textwidth,
%	tick label style={/pgf/number format/fixed,
%		/pgf/number format/precision=4}
%	]
%	\foreach \n in {\C,\B,...,1} {
%		\pgfplotstablegetcolumnnamebyindex{\n}\of{\ventanascdsgaussian}\to{\colname}
%		\pgfplotstablegetelem{0}{\colname}\of{\colorbyfilter}
%		\let\color=\pgfplotsretval
%		\pgfplotstablegetelem{1}{\colname}\of{\colorbyfilter}
%		\let\marca=\pgfplotsretval
%		\edef\temp{
%			\noexpand\addplot+[\color, mark=\marca, solid, mark options={mark size = 1.5, fill=\color!70}] 
%		}
%		\temp	
%		table[x=p, y index=\n]{\ventanascdsgaussian};
%		
%		\addlegendentryexpanded{\colname}
%	}%
%	\pgfplotstablegetelem{\R}{[index]1}\of{\ventanascdsgaussian}
%	\let\ytozoom=\pgfplotsretval
%	\coordinate (spypoint) at (axis cs:0.165,\ytozoom);
%	\coordinate (spyviewer) at (axis cs:0.1,22000000);
%	\spy on (spypoint) in node [fill=white] at (spyviewer);
%	\end{axis}
%	\end{tikzpicture}
%	\caption{Ruido gaussiano. CDS por $\sigma^2$}
%	\label{fig:gaussian_cds}
%\end{figure*}
%
%
%\begin{table*}
%	\small
%	\centering
%	\caption{Ruido gaussiano. $CDS \times 10^7$ por $\sigma^2$}
%	\pgfkeys{/pgf/number format/precision=5}
%	\pgfplotstabletypeset[
%	col sep=tab,
%	header=has colnames,
%	every first column/.style={
%		string type,
%		column type/.add={|}{|}
%	},
%	every head row/.style={
%		before row=\hline,
%		after row=\hline
%	},
%	every last column/.style={column type/.add={}{|}},
%	every last row/.style={
%		after row=\hline,
%	},
%	every row 0 column 1/.style={
%		postproc cell content/.style={
%			@cell content/.add={$\bf}{$}
%		}
%	},
%	every row 0 column 2/.style={
%		postproc cell content/.style={
%			@cell content/.add={$\bf}{$}
%		}
%	},
%	every row 0 column 3/.style={
%		postproc cell content/.style={
%			@cell content/.add={$\bf}{$}
%		}
%	},
%	every row 0 column 4/.style={
%		postproc cell content/.style={
%			@cell content/.add={$\bf}{$}
%		}
%	},
%	every row 0 column 5/.style={
%		postproc cell content/.style={
%			@cell content/.add={$\bf}{$}
%		}
%	},
%	every row 0 column 6/.style={
%		postproc cell content/.style={
%			@cell content/.add={$\bf}{$}
%		}
%	},
%	every row 1 column 7/.style={
%		postproc cell content/.style={
%			@cell content/.add={$\bf}{$}
%		}
%	},
%	every row 1 column 8/.style={
%		postproc cell content/.style={
%			@cell content/.add={$\bf}{$}
%		}
%	},
%	every row 1 column 9/.style={
%		postproc cell content/.style={
%			@cell content/.add={$\bf}{$}
%		}
%	},
%	my multistyler={1,...,17}{
%		divide by={10000000}
%	},
%	]{anexos/files/ventanas_cds_gaussian_tabla.txt}
%	\label{tabla:gaussian_cds}
%\end{table*}
%
%%cds sal y pimienta
%\pgfplotstableread[
%col sep = tab,
%header=has colnames,
%comment chars={P},
%]
%{anexos/files/ventanas_cds_salt_and_pepper.txt}{\ventanascdssaltandpepper}
%
%
%\begin{figure*}
%	\centering
%	\begin{tikzpicture}[spy using outlines={circle, magnification=9.5, size=3.5cm, connect spies}]
%	\begin{axis}[
%	xmax = 0.175,
%	legend style={mark options={scale=1.5}},
%	xlabel={$p$},
%	ylabel={CDS},
%	legend pos=outer north east,
%	height=0.5\textwidth,
%	width=0.6\textwidth,
%	tick label style={/pgf/number format/fixed,
%		/pgf/number format/precision=4}
%	]
%	\foreach \n in {\C,\B,...,1} {
%		\pgfplotstablegetcolumnnamebyindex{\n}\of{\ventanascdssaltandpepper}\to{\colname}
%		\pgfplotstablegetelem{0}{\colname}\of{\colorbyfilter}
%		\let\color=\pgfplotsretval
%		\pgfplotstablegetelem{1}{\colname}\of{\colorbyfilter}
%		\let\marca=\pgfplotsretval
%		\edef\temp{
%			\noexpand\addplot+[\color, mark=\marca, solid, mark options={mark size = 1.5, fill=\color!70}] 
%		}
%		\temp	
%		table[x=p, y index=\n]{\ventanascdssaltandpepper};
%		
%		\addlegendentryexpanded{\colname}
%	}%
%	\pgfplotstablegetelem{\R}{MINW9}\of{\ventanascdssaltandpepper}
%	\let\ytozoom=\pgfplotsretval
%	\coordinate (spypoint) at (axis cs:0.165,\ytozoom);
%	\coordinate (spyviewer) at (axis cs:0.137,10170000);
%	\spy on (spypoint) in node [fill=white] at (spyviewer);
%	\end{axis}
%	\end{tikzpicture}
%	\caption{Ruido sal y pimienta. CDS por $p$}
%	\label{fig:salt_and_pepper_cds}
%\end{figure*}
%
%
%\begin{table*}
%	\centering
%	\small
%	\caption{Ruido sal y pimienta. $CDS \times 10^7$ por $p$}
%	\pgfkeys{/pgf/number format/precision=4}
%	\pgfplotstabletypeset[
%	col sep=tab,
%	header=has colnames,
%	every first column/.style={
%		string type,
%		column type/.add={|}{|},
%	},
%	every head row/.style={
%		before row=\hline,
%		after row=\hline
%	},
%	every last column/.style={column type/.add={}{|}},
%	every last row/.style={
%		after row=\hline,
%	},
%	every row 0 column 1/.style={
%		postproc cell content/.style={
%			@cell content/.add={$\bf}{$}
%		}
%	},
%	every row 0 column 2/.style={
%		postproc cell content/.style={
%			@cell content/.add={$\bf}{$}
%		}
%	},
%	every row 0 column 3/.style={
%		postproc cell content/.style={
%			@cell content/.add={$\bf}{$}
%		}
%	},
%	every row 0 column 4/.style={
%		postproc cell content/.style={
%			@cell content/.add={$\bf}{$}
%		}
%	},
%	every row 0 column 5/.style={
%		postproc cell content/.style={
%			@cell content/.add={$\bf}{$}
%		}
%	},
%	every row 0 column 6/.style={
%		postproc cell content/.style={
%			@cell content/.add={$\bf}{$}
%		}
%	},
%	every row 0 column 7/.style={
%		postproc cell content/.style={
%			@cell content/.add={$\bf}{$}
%		}
%	},
%	every row 0 column 8/.style={
%		postproc cell content/.style={
%			@cell content/.add={$\bf}{$}
%		}
%	},
%	every row 0 column 9/.style={
%		postproc cell content/.style={
%			@cell content/.add={$\bf}{$}
%		}
%	},
%	my multistyler={1,...,17}{
%		divide by={10000000}
%	},
%	]{anexos/files/ventanas_cds_salt_and_pepper_tabla.txt}
%	\label{tabla:salt_and_pepper_cds}
%\end{table*}
%
%%cds speckle
%\pgfplotstableread[
%col sep = tab,
%header=has colnames,
%comment chars={P},
%]
%{anexos/files/ventanas_cds_speckle.txt}{\ventanascdspeckle}
%
%
%\begin{figure*}
%	\centering
%	\begin{tikzpicture}[spy using outlines={circle, magnification=9.5, size=3.5cm, connect spies}]
%	\begin{axis}[
%	xmax = 0.175,
%	legend style={mark options={scale=1.5}},
%	xlabel={$\sigma^2$},
%	ylabel={CDS},
%	legend pos=outer north east,
%	height=0.5\textwidth,
%	width=0.6\textwidth,
%	tick label style={/pgf/number format/fixed,
%		/pgf/number format/precision=4}
%	]
%	\foreach \n in {\C,\B,...,1} {
%		\pgfplotstablegetcolumnnamebyindex{\n}\of{\ventanascdspeckle}\to{\colname}
%		\pgfplotstablegetelem{0}{\colname}\of{\colorbyfilter}
%		\let\color=\pgfplotsretval
%		\pgfplotstablegetelem{1}{\colname}\of{\colorbyfilter}
%		\let\marca=\pgfplotsretval
%		\edef\temp{
%			\noexpand\addplot+[\color, mark=\marca, solid, mark options={mark size = 1.5, fill=\color!70}] 
%		}
%		\temp	
%		table[x=p, y index=\n]{\ventanascdspeckle};
%		
%		\addlegendentryexpanded{\colname}
%	}%
%	\pgfplotstablegetelem{\R}{SMOW9}\of{\ventanascdspeckle}
%	\let\ytozoom=\pgfplotsretval
%	\coordinate (spypoint) at (axis cs:0.165,\ytozoom);
%	\coordinate (spyviewer) at (axis cs:0.12,20000000);
%	\spy on (spypoint) in node [fill=white] at (spyviewer);
%	\end{axis}
%	\end{tikzpicture}
%	\caption{Ruido speckle. CDS por $\sigma^2$}
%	\label{fig:speckle_cds}
%\end{figure*}
%
%
%\begin{table*}
%	\centering
%	\small
%	\caption{Ruido speckle. $CDS \times 10^7$ por $\sigma^2$}
%	\pgfkeys{/pgf/number format/precision=5}
%	\pgfplotstabletypeset[
%	col sep=tab,
%	header=has colnames,
%	every first column/.style={
%		string type,
%		column type/.add={|}{|},
%	},
%	every head row/.style={
%		before row=\hline,
%		after row=\hline
%	},
%	every last column/.style={column type/.add={}{|}},
%	every last row/.style={
%		after row=\hline,
%	},
%	every row 0 column 1/.style={
%		postproc cell content/.style={
%			@cell content/.add={$\bf}{$}
%		}
%	},
%	every row 0 column 2/.style={
%		postproc cell content/.style={
%			@cell content/.add={$\bf}{$}
%		}
%	},
%	every row 0 column 3/.style={
%		postproc cell content/.style={
%			@cell content/.add={$\bf}{$}
%		}
%	},
%	every row 0 column 4/.style={
%		postproc cell content/.style={
%			@cell content/.add={$\bf}{$}
%		}
%	},
%	every row 0 column 5/.style={
%		postproc cell content/.style={
%			@cell content/.add={$\bf}{$}
%		}
%	},
%	every row 0 column 6/.style={
%		postproc cell content/.style={
%			@cell content/.add={$\bf}{$}
%		}
%	},
%	every row 0 column 7/.style={
%		postproc cell content/.style={
%			@cell content/.add={$\bf}{$}
%		}
%	},
%	every row 0 column 8/.style={
%		postproc cell content/.style={
%			@cell content/.add={$\bf}{$}
%		}
%	},
%	every row 0 column 9/.style={
%		postproc cell content/.style={
%			@cell content/.add={$\bf}{$}
%		}
%	},
%	my multistyler={1,...,17}{
%		divide by={10000000}
%	},
%	]{anexos/files/ventanas_cds_speckle_tabla.txt}
%	\label{tabla:speckle_cds}
%\end{table*}
\normalsize

En la Figura \ref{fig:gaussian_mae} se observan las curvas de tendencia de los filtros aplicados sobre imágenes contaminadas con ruido gaussiano. Como puede observarse, para el ruido gaussiano, el mejor filtro resultó la propuesta utilizando SMO, MAX, VAR y MEAN para el cálculo de vector de pesos. El Cuadro \ref{tabla:gaussian_mae} contiene  la suma total de los puntos en la curva de tendencia por método de ordenamiento.

La Figura \ref{fig:salt_and_pepper_mae} muestra las curvas de tendencia de los filtros aplicados sobre imágenes contaminadas con ruido sal y pimienta. En este caso el filtro que obtuvo el menor promedio general y cuya curva se ubicó debajo de todas las demás fue ED. Los filtros MAX, SMO, VAR y MEAN obtuvieron mejor rendimiento después de ED. En el Cuadro \ref{tabla:salt_and_pepper_mae} contiene  la suma total de los puntos en la curva de tendencia por método de ordenamiento.

En la Figura \ref{fig:speckle_mae} se encuentran las curvas de tendencia de los filtros aplicados sobre imágenes contaminadas con ruido speckle. Los filtros SMO, MAX y VAR obtuvieron los mejores resultados. El filtro SMOW9 obtuvo los mejores resultados para todos los puntos de la curva. El filtro ED resultó el mejor filtro del estado del arte. El Cuadro \ref{tabla:speckle_mae} contiene  la suma total de los puntos en la curva de tendencia por método de ordenamiento ordenados de manera ascendente.

%Desde la Figura \ref{fig:gaussian_cds} en adelante se ilustran los resultados para la métrica CDS. La Figura \ref{fig:gaussian_cds} muestra los resultados para el ruido gaussiano en la mencionada métrica. Los mejores filtros resultaron ser nuevamente los propuestos (SMO, MAX, VAR, MEAN) excepto MO1, MO2 y MIN que fueron superados por el mejor filtro del estado del arte, ED. En el zoom se hace visible la diferencia entre los filtros propuestos y el filtro ED. El Cuadro \ref{tabla:gaussian_cds} proporciona los valores numéricos a partir de los cuales se formaron las gráficas mencionadas; en el mismo puede observarse que el filtro SMOW9 obtuvo los mejores resultados hasta el punto $0.105$, desde el punto $0.125$ en adelante MAXW9 obtuvo los mejores resultados.
%
%La Figura \ref{fig:salt_and_pepper_cds} proporciona los resultados de la métrica CDS para el ruido sal y pimienta. Nuevamente, tal y como ocurrió con la métrica MAE, los filtros propuestos no resultaron tan eficaces para este ruido. En la figura se puede ver la marcada ventaja que el filtro DLAB posee por sobre los demás. El Cuadro \ref{tabla:gaussian_cds} brinda los valores numéricos utilizados en la construcción de las curvas mencionadas. Cabe resaltar que para la métrica MAE el ordenamiento ED fue el mejor filtro, no así  en la métrica CDS, que resultó ser el filtro con ordenameinto DLAB.  
%
%Por último, la Figura \ref{fig:speckle_cds} posee los valores de la métrica CDS para filtros aplicados sobre imágenes contaminadas con el ruido speckle. El Cuadro \ref{tabla:speckle_cds} posee los resultados numéricos de dichos gráficos de curvas. En este caso se puede observar nuevamente un solapamiento de las curvas correspondientes a los filtros propuestos (MAX, SMO, VAR, MEAN), pero a una distancia evidente de las curvas correspondientes al estado del arte, de entre las cuales el mejor fue ED. El filtro SMOW9 obtuvo los mejores resultados en todos los puntos de $\sigma^2$.

Las dos siguientes aplicaciones utilizan matemática morfológica. Los operadores no son puramente morfológicos, ya que no se puede garantizar sus propiedades teóricas, como la idempotencia para la apertura y el cierre. En la Figura \ref{fig:ContraEjemplo} se puede observar un contra-ejemplo, donde el operador apertura no es idempotente ($f\circ B \neq (f\circ B) \circ B$). A la imagen sintética en la Figura \ref{fig:ContraEjemplo}(a) se le aplica una apertura con un elemento estructurante 3 x 3, donde el dominio $D$ de la imagen es el propio vecindario del elemento estructurante, y la función aplicada a cada componente de color es el promedio dentro de $D$, dando como resultado la imagen en la Figura \ref{fig:ContraEjemplo}(b). A la imagen de la Figura \ref{fig:ContraEjemplo}(b) se le aplica de nuevo el operador apertura con el mismo elemento estructurante y donde el dominio $D$ es también el propio vecindario del elemento estructurante, dando como resultado la imagen en la Figura  \ref{fig:ContraEjemplo}(c). La imagen resultante no es igual a la anterior, esto se puede visualizar en la imagen \ref{fig:ContraEjemplo}(c), que es el resultado de realizar la diferencia entre ambas imágenes. De esta manera se demuestra que el operador apertura no es idempotente, sucediendo lo mismo para el operador cierre.


\begin{figure*}
	\makebox[\linewidth][c]{%
		\begin{subfigure}[b]{0.25\textwidth}
			\centering
			\includegraphics[width=0.95\linewidth]{fig/ImgCortes.png}
			\caption{Imagen sintética}
			\label{fig:ImaSin}
		\end{subfigure}
		\begin{subfigure}[b]{0.25\textwidth}
			\centering
			\includegraphics[width=0.95\linewidth]{fig/Apertura1.png}
			\caption{Apertura de la imagen (a)}
			\label{fig:Apert1}
		\end{subfigure}
		\begin{subfigure}[b]{0.25\textwidth}
			\centering
			\includegraphics[width=0.95\linewidth]{fig/Apertura2.png}
			\caption{Apertura de la imagen (b)}
			\label{fig:Apert2}
		\end{subfigure}
		\begin{subfigure}[b]{0.25\textwidth}
			\centering
			\includegraphics[width=0.95\linewidth]{fig/DiferenciaApertura.png}
			\caption{Diferencia entre (b) y (c)}
			\label{fig:Apert1}
		\end{subfigure}
	}\\
	\caption{Un contra-ejemplo que demuestra que la apertura no es idempotente con el ordenamiento propuesto.}	
	\label{fig:ContraEjemplo}
\end{figure*}

Esto se debe que un color puede ser mayor o menor a otro color en un dominio $D$, pero no serlo en otro dominio, debido a que la información extraida en forma de pesos (resultado de aplicarle una función) puede ser distinta. Incluso en el mismo dominio pero en la siguiente iteración (producto de aplicarle de nuevo el mismo operador) pueden varian los pesos, ya que también la información extraida varia de una iteración a otra. En la literatura estos operadores son llamados pseudo-operadores \cite{hanbury2001morphological,aptoula2007pseudo,aptoula2008alpha,angulo2010pseudo,chen2002pseudo}. 

La transformada top-hat es ampliamente utilizada en diferentes aplicaciones \cite{soille2013morphological,mukhopadhyay2000multiscale,soille1997note,bai2010analysis,bai2010infrared,bai2010analysis1}. Como mencionamos anteriormente la transformada top-hat blanca extrae las regiones brillantes de la imagen y la transformada top-hat extrae las zonas oscuras de la imagen

\subsection{Aplicación 2: Mejoramiento de contraste}
 Una idea b\'asica de mejoramiento de contraste  de la imagen $f$ es a\~nadir las regiones brillantes de la imagen $f$ y sustraer las regiones oscuras de la imagen $f$ como sigue \cite{soille2013morphological}:
\begin{equation}
\label{contraste}
Contraste(f) = f + WTH(f) - BTH(f) 
\end{equation}


La efectividad de la aplicación del mejoramiento del contraste se determina utilizando el método denominado Color Enhancement Factor (CEF) que cuantifica el nivel de mejora del contraste de una imagen cómo se menciona en \cite{susstrunk2003color}. Este método aplicado a la imagen $f$ esta basado en la media y la desviación estandar de dos ejes de una sencilla representación de color contrario con $\gamma = f_1 - f_2$ y $\beta = \frac{1}{2}(f_1 + f_2) - f_3$.
La ecuación \ref{eq:cef} representa el nivel de mejora del contraste de la imagen $f$ de la siguiente forma:

\begin{equation}
CM(f)=\sqrt{\sigma_{\gamma}^{2} + \sigma_{\beta}^{2}} + \sqrt{\mu_{\gamma}^{2} + \mu_{\beta}^{2}}
\label{eq:cef}
\end{equation} Donde $\sigma_{\gamma}$ y $\sigma_{\beta}$ corresponden a la desviación estándar  de $\gamma$ y $\beta$ respectivamente. De manera similar, $\mu_{\gamma}$ y $\mu_{\beta}$ corresponde a la media respectivamente.

Entonces el CEF se calcula por medio de la razón entre la imagen $f'$ y la imagen original $f$:

\begin{equation}
CEF = \frac{CM(f^{'})}{CM(f)}
\label{cef}
\end{equation}

Donde $CM(f^{'})$ es el valor obtenido de la imagen contrastada $f'$ producto de aplicar la ecuación \ref{eq:cef} y $CM(f)$ representa el resultado de aplicar la ecuación \ref{eq:cef} a la imagen original $f$. Si el resultado es $> 1$ entonces la métrica de la ecuación \ref{cef} indica una mejora en el contraste, de lo contrario, la métrica indica que no hay una mejora del contraste.



\subsubsection{Resultados}
 Las pruebas fueron realizadas con 100 im\'agenes de prueba de \cite{arbelaez2007berkeley} y se utilizaron las mismas abreviaciones que en el experimento anterior para diferenciar los métodos de ordenamiento con diferente descomposición de dominio. 
En el Cuadro \ref{tab:EXP_N1} se puede ver los resultados de las distintas iteraciones (iter) del algoritmo de mejora de contraste (aplicar varias veces la ecuación \ref{contraste} a la misma imagen).  
Se observa que SMOB tiene mejores resultados en todas la iteraciones, le sigue la Varianza VARB y luego siguen los demas métodos. La mejora utilizando SMOB a medida que crece la cantidad de iteraciones es aproximademente del 3\% y la diferencia con el segundo y el tercero del 0,30\% y 2,25\% en promedio en cada una de la iteraciones.
Se puede constatar que a medida que se aumenta la cantidad de iteraciones  tambien mejora más el contraste según la métrica mencionanda, siendo también importante el dominio y el método de ordenamiento utilizado.

\begin{table}
\caption{ Mejora de Contraste}
\label{tab:EXP_N1}
\begin{tabular}{ccccccccc}
\hline METODO& iter1&iter2&iter3&iter4\\
\hline  \textbf{SMOB}& \textbf{1,03482}& \textbf{1,03482}& \textbf{1,06084}& \textbf{1,08889}\\
\hline  \textbf{VARB}& \textbf{1,0329}& \textbf{1,0329}& \textbf{1,05798}& \textbf{1,0852}\\
\hline MO2W9&1,02047&1,02047&1,03668&1,05426\\
\hline MO1W9&1,02045&1,02045&1,03664&1,05421\\
\hline SMOW9&1,0203&1,0203&1,03639&1,05369\\
\hline MAXW9&1,02023&1,02022&1,03629&1,05364\\
\hline BM&1,01997&1,01997&1,03576&1,05273\\
\hline MINW9&1,01992&1,01991&1,03573&1,05295\\
\hline ED&1,01989&1,01989&1,0356&1,05265\\
\hline AMLEX&1,01988&1,01987&1,0352&1,05159\\
\hline MEANW9&1,01986&1,01986&1,03567&1,05277\\
\hline MO1B&1,01937&1,01937&1,03497&1,05197\\
\hline MAXB&1,01928&1,01928&1,03493&1,05199\\
\hline MO2B&1,01922&1,01922&1,03481&1,05182\\
\hline MEANB&1,01912&1,01912&1,03461&1,0514\\
\hline LEX&1,01909&1,01909&1,03369&1,04908\\
\hline ALEX&1,01909&1,01909&1,03369&1,04908\\
\hline MINB&1,01908&1,01908&1,03448&1,05128\\
\hline DLAB&1,01756&1,01756&1,03049&1,04426\\
\hline HLEX&1,00743&1,00743&1,0128&1,01863\\
\hline VARW3&0,99727&0,99725&0,98268&0,98489\\

\hline

\end{tabular}
\end{table}

En la Figura \ref{fig:mejora} se puede ver un ejemplo de mejora de contraste de una imagen, a la cual se le aplicó 4 iteraciones de la ecuación \ref{contraste} con el método de ordenamiento propuesto utilizando SMO para el cálculo de los pesos. La imagen resultante claramente es mucho más contrastada que la imagen original. 
 
%\begin{figure}
%    \centering
%    \subfigure[Imagen original]{\includegraphics[width=73mm]{fig/Imagen3.jpg}}
%		\subfigure[Imagen Mejorada]{\includegraphics[width=73mm]{fig/Resultado3_3_4.jpg}}
%		  \caption{(a) Imagen con mejora de contraste aplicando 4 veces la ecuación \ref{contraste} con el método de ordenamiento propuesto}
%  \label{fig:mejora}
%\end{figure}

\begin{figure}
	\makebox[\linewidth][c]{%
		\begin{subfigure}[b]{0.5\textwidth}
			\centering
			\includegraphics[width=0.8\textwidth]{fig/Imagen3.jpg}
			\caption{Imagen original}
			\label{fig:imagen_original_3}
		\end{subfigure}
	}\\
	\makebox[\linewidth][c]{%
		\begin{subfigure}[b]{0.5\textwidth}
			\centering
			\includegraphics[width=0.8\textwidth]{fig/Resultado3_3_4.jpg}
			\caption{Imagen Mejorada}
			\label{fig:imagen_mejorada_3}
		\end{subfigure}
	}\\
	\caption{(a) Imagen con mejora de contraste aplicando 4 veces la ecuación \ref{contraste} con el método de ordenamiento propuesto}
  \label{fig:mejora}
\end{figure}


\subsection{Aplicación 3: Clasificación de texturas}

 El problema de caracterización y clasificación de textura consta de dos pasos. En una primera instancia, se calculan características de una imagen que permitan describir numéricamente sus propiedades de textura por medio de un vector de características o descriptor. Posteriormente es asignada una clase de textura de acuerdo criterios de similaridad entre los descriptores \cite{hanbury2005illumination}.

La granulometría y la covarianza morfológica son las principales herramientas morfológicas de caracterización de textura, ambas utilizan distribuciones de intensidad para describir las propiedades de las texturas \cite{lefevre2009beyond}.

%Los descriptores de textura morfológicos tienen la capacidad de describir efectivamente la regularidad y la direccionalidad por medio de la granulometría y la covarianza morfológica  \cite{hanbury2005illumination}. 

La manera en que la información de color y textura es incorporada al descriptor es estudiada en \cite{palm2004color,van2005parallel}. En este trabajo las herramientas morfológicas utilizan el enfoque integrativo donde la información de color y textura son procesadas conjuntamente.

La granulometría fue propuesta en \cite{matheron1975random} y es aplicado en la extracción de características y la estimación de tamaño \cite{vincent2000granulometries,soille2013morphological}. Consiste en una familia de aperturas $f \,\circ \, \lambda B $ de $n+1$ elementos incluyendo la imagen de entrada. Está parametrizada por el tamaño $\lambda$ creciente del elemento estructurante ($0 \leq \lambda \leq n $). Los valores son recolectados por una medida de evaluación que usualmente es el volumen ($\mathrm{Vol}$):

\begin{equation}
G^{n}_{j}(f,\lambda)= \mathrm{Vol}([f\circ \lambda B ]_{j}) \; / \; \mathrm{Vol}(f_{j})
\end{equation}

Donde $j$ es el j-ésimo componente de la imagen a color $f$ y el volumen está definido como:
\begin{equation}
\mathrm{Vol}(f_j) = \sum_{\substack{u\in \{1, ..., M\}\\ v \in \{1, ..., N\}}}[f(u,v)]_{j} 
\end{equation}

La covarianza morfológica propuesta en \cite{matheron1975random,maragos1989pattern} denotada por $K$ de una imagen $f$, está definida como el volumen de la imagen $f$, luego de aplicarse la erosión $\varepsilon$ a partir de un par de pixeles $(u,v)$ y  $(u',v')$ separados por un vector $\vec{v}$ denotado por $P_{2,\vec{v}}$.\\
En la práctica $K $ es calculado aplicando la erosión $\varepsilon$ a la imagen original $f$ con el elemento estructurante $P_{2,\vec{v}}$ variando orientaciones y longitudes de $\vec{v}$, donde $n$ es el número de variaciones de $\vec{v}$. Su versión normalizada está dada por :
\begin{equation}
K^{n}_{j}(f,P_{2,\vec{v}})  = \mathrm{Vol}([\varepsilon(f,P_{2,\vec{v}})]_{j}) / \mathrm{Vol}(f_{j})
\end{equation}
Permite obtener una distribución de orientación y distancia de una imagen de textura \cite{aptoula2007morphological}.

El método de ordenamiento utilizado para dar soporte a las herramientas morfológicas de caracterización de textura, inciden en los porcentajes de clasificación. Esto es debido a que las distribuciones de intensidad utilizadas como descritores de textura, varían de acuerdo a las imágenes intermedias. Estas imágenes intermedias son resultado de la aplicación de un filtro morfológico.

%--------------------------------------------------------
\subsubsection{Resultados}
\label{sec:resultadosexperi}
Las pruebas fueron realizadas con la base de datos OutexTC13 compuesta por 1360 imágenes de dimensiones 128 $\times $ 128 píxeles, con 68 clases de texturas de superficies (Figura \ref{fig:outex13}) con 20 muestras de cada clase, donde el 50\% de cada clase es el conjunto de entrenamiento. Totalizan 680 imágenes de entrenamiento y prueba respectivamente \cite{ojala2002outex}.  
El clasificador utilizado fue el k-nn (k-nearest neighbors) utilizando distancia euclídea con k=1. 

\begin{figure}
	\centering
		\includegraphics[scale=0.25]{fig/outex13.png}
	\caption{Muestras de textura OutexTC13}
	\label{fig:outex13}
\end{figure}

La finalidad del experimento fue obtener los porcentajes de clasificación utilizando los métodos de ordenación expuestos.
Se evaluaron varios parámetros estadísticos en la estrategia de orden propuesta en este trabajo.
%---configuracion de la granulometria
En las pruebas de granulometría se han utilizado elementos estructurantes de forma cuadrada de tamaño $\lambda$ y de lado $2\lambda+1$ píxeles, variando $\lambda$  de 1 a 15. Para cada elemento de la serie se calculan 15 valores para cada canal que posteriormente se concatenan. La elección del incremento simple de $\lambda$, está basado en que los incrementos menores proveen mejores resultados de clasificación \cite{de2006selecting}. 
Con respecto a la configuración de los parámetros de la propuesta, cada muestra de textura fue dividida en $2\times2$ sub-regiones y denotado por el sufijo W4. Esta división permite que cada muestra de textura sea dividida en partes iguales y perceptualmente similares. Se denota con el sufijo B cuando el dominio de la imagen es el propio elemento estructurante. 

%---configuracion de la covarianza morfologica
La covarianza morfológica requiere la variación de la dirección y la distancia entre el par de puntos que componen el elemento estructurante. Las direcciones utilizadas fueron $0^{\circ}$, $45^{\circ}$, $90^{\circ}$,  $135^{\circ}$, en la práctica sólo estas direcciones son de importancia y perceptiblemente reconocibles \cite{hanbury2005illumination}. Las distancias de separación utilizadas para cada dirección fueron desde 1 a 20 píxeles. Utilizando estas 4 direcciones y las 20 distancias se han generado 80 valores para cada canal, finalmente estos valores son concatenados para obtener el vector de características de la muestra de textura. 
 
\begin{table}
\caption{ Resultados de clasificación por Métodos de Ordenamiento}
\label{tab:experiment1_t1}

\begin{tabular}{@{}lrr@{}}
\toprule
&\multicolumn{2}{c}{ \% Clasificados correctamente }\\
\cline{2-3}
\textbf{Ordenación } & \textbf{Covarianza} &\textbf{Granulometría } \\ 
\cmidrule{1-3}
%Marginal  & 85,78 & 88,68 \\ \hline
BM & 79,56 & 77,79 \\ \hline
ED & 81,91 & \textbf{84,11}  \\ \hline
LEX & 79,74 & 80,15 \\ \hline
ALEX & 76,32   &  69,12  \\ \hline
AMLEX &  81,62   &  81,30  \\ \hline
HLEX &  \textbf{83,97}  &  77,35  \\ \hline
DLAB  & \textbf{84,26}  & 72,35   \\ \hline
MEANW4 & 81,47 & 78,97 \\ \hline %$\mathrm{HST_{2}} $ Promedio 
%$\mathrm{HST_{2}} $ Entropía & 82,65 & \textbf{85,88} \\ \hline
SMOW4 & 82,50 & \textbf{85,44} \\ \hline %$\mathrm{HST_{2}} $
MO1W4 & 81,76 & 82,65 \\ \hline %$\mathrm{HST_{2}} $ Moda Min  
MO2W4 & 80,44 & 80,01 \\ \hline %$\mathrm{HST_{2}} $ Moda Max  
MAXW4 & 81,62 & 81,32 \\ \hline % $\mathrm{HST_{2}} $ Máximo
MINW4 & 81,47 & 83,38 \\ \hline %$\mathrm{HST_{2}} $ Mínimo 
VAR   & 81,76 &  83,97  \\ \hline 
MEANB & 81,91 & 78,82 \\ \hline %$\mathrm{HST}_{ b} $ Promedio
SMOB & \textbf{83,82} & \textbf{84,71} \\ \hline %$\mathrm{HST}_{b} $ Suavidad
MO1B & 82,01  & 83,23  \\ \hline %$\mathrm{HST_{2}} $ Moda Min  
MO2B & 81,21 &  81,36  \\ \hline %$\mathrm{HST_{2}} $ Moda Max  
MAXB & 81,32 & 81,06 \\ \hline %$\mathrm{HST}_{ b} $ 
MINB & 81,05 & 83,12 \\ \hline %$\mathrm{HST}_{ b} $
VARB & 83,09 &  82,21  \\ \hline 
%$\mathrm{HST_{2}} $ Desviación  & 82,01 & \textbf{85,02} \\ \hline
%$\mathrm{HST_{2}} $ Asimetria  & 81,91 & 83,09 \\ \hline
%$\mathrm{HST_{2}} $ Uniformidad & 80,74 & 80,44 \\ \hline
%$\mathrm{HST_{2}} $ Momento muestral & 79,56 & 60,88 \\ \hline
%$\mathrm{HST}_{ b} $ Desviación & \textbf{82,94} & \textbf{85,29} \\ \hline
%$\mathrm{HST}_{ b} $ Asimetría & \textbf{83,24} & \textbf{84,85} \\ \hline
%$\mathrm{HST_{2}} $ Energía & 82,35 & 79,56 \\ \hline
%$\mathrm{HST}_{ b} $ Entropía & 82,21 & 64,85 \\ \hline
%$\mathrm{HST}_{ b} $ Uniformidad & 80,59 & 80,29 \\ \hline
%$\mathrm{HST}_{ b} $ Momento Muestral & 46,47 & 62,78 \\ \hline
%$\mathrm{HST}_{ b} $ Energía & 82,03 & 81,47 \\ \hline
%$\mathrm{HST}_{ 2,4,8} $ Promedio & 81,62 & 84,12 \\ \hline
%$\mathrm{HST}_{ 2,4,8} $ Entropia & 82,79 & \textbf{85,74} \\ \hline
%$\mathrm{HST}_{ 2,4,8} $ Asimetría & 81,91 & 82,65 \\ \hline
%$\mathrm{HST}_{ 2,4,8} $ Desviación & 82,50 & 84,41 \\ \hline
%$\mathrm{HST}_{ 2,4,8} $ Uniformidad  & 80,74 & 80,59 \\ \hline
%$\mathrm{HST}_{ 2,4,8} $ Suavidad  & 82,50 & 84,56 \\ \hline
%$\mathrm{HST}_{ 2,4,8} $ Momento Muestral & 79,12 & 60,88 \\ \hline
%$\mathrm{HST}_{ 2,4,8} $ Moda  & 81,76 & 83,09 \\ \hline
%$\mathrm{HST}_{ 2,4,8} $ Máximo & 81,62 & 81,32 \\ \hline
%$\mathrm{HST}_{ 2,4,8} $ Mínimo & 81,47 & 83,53 \\ \hline
%$\mathrm{HST}_{ 2,4,8} $ Energía & 82,35 & 75,88 \\ \hline
\end{tabular}
\label{exper_ordenaciones}
\end{table}
%-  - - - - - - - - - - - - - -
En el Cuadro \ref{tab:experiment1_t1} los tres mejores resultados de cada método fueron marcados en negrita. La Covarianza Morfológica con los ordenamientos HLEX y DLAB tienen rendimientos superiores de $\approx 1\%$ con respecto a SMOW4 que presenta el mejor rendimiento en el espacio RGB. Los resultados son consistentes con los experimentos realizados en \cite{hanbury2005illumination} donde se obtienen mejores resultados en el espacio L*a*b en comparación al espacio RGB utilizando elementos estructurantes de longitud variante. Esto se debe a que en esos espacios de color la información cromática está separada de la luminosidad o intensidad. En experimentos mencionados en \cite{bianconi2011theoretical} sobre la percepción de la textura indican que la textura y el color se perciben de manera independiente. 
Con la granulometría el espacio RGB presenta mejores resultados con los ordenamientos SMOW4 y SMOB, ambos superiores al ordenamiento ED en ($\approx 1,33 \% $). Por otra parte el ordenamiento ED provee resultados muy superiores a los ordenamientos BM, LEX, ALEX y AMLEX.

Los ordenamientos SMOW4 y SMOB (85,44\%, 84,71\%) muestran los mejores resultados de clasificación (con granulometría) con respecto a los métodos de ordenamiento del estado del arte implementados. La división en $2\times2$ sub-regiones (W4) conducen a mejores resultados que usando B.  

\section{Conclusiones y Trabajo Futuro}

En este trabajo se presenta una nueva estrategia de ordenamiento de colores RGB que es dependiente de la imagen. El ordenamiento se realiza por medio de extraer información del histograma de cada componente de color en cierto dominio de la imagen. Se presentaron dos estrategias de descomposición de dominios para extraer dicha información, una es extrayendo información de la misma ventana del filtro, y otra consiste en dividir la imagen en sub-regiones de mismo tamaño, tomando la unión de ellos para cuando la ventana del filtro toma más de una sub-región.
Fueron realizadas pruebas para 3 aplicaciones de procesamiento de imágenes: Eliminación de ruido, estiramiento de contraste y caracterización texturas para su posterior clasificación.  Para estas dos últimas aplicaciones se utiliza matemática morfológica. Los operadores morfologicos en este caso son pseudo-operadores, ya que se no puede garantizar ciertas propiedades teóricas, como la idempotencia. 
El filtro mediana fue utilizado para la eliminación de ruido, consiguiendo mejores resultados con algunos pesos extraidos a diferentes métodos del estado del arte, tanto para ruido gaussiano, como speckle. Para el ruido sal y pimienta la distancia euclidiana al origen en el espacio de color RGB en la métrica MAE  dió mejores resultados al método propuesto con las diferentes informaciones extraidas de cada componente.  Esto se debe a que el ruido sal está expresada por el valor mínimo en cada componente de color, y el ruido pimienta está expresado como el valor máximo en cada componente. De esta manera, si los colores de los pixeles se ordena por la distancia euclidiana, es bastante improbable que el filtro mediana seleccione un pixel que sea ruido. 
Para la aplicación de mejora de contraste el método propuesto se mostró más eficiente según la métrica CEF con diferentes pesos extraidos a partir de información de la imagen. Para la caracterización de texturas utilizando la Covarianza Morfológica con el ordenamiento lexicográfico  I→H→S \cite{ortiz2004gaussian} y la distancia euclidiana al color $(0,0,0)$ en el espacio de color L*a*b* \cite{ortiz2002procesamiento} tienen mejores rendimientos a la propuesta, demostrado en su mejor clasificación. El método propuesto utilizando la suavidad como peso para cada componente  consigue mejores resultados  utilizando la granulometría como caracterización de texturas.
Como trabajo futuro se propone hacer un análisis de la importancia de descomposición de dominios para extraer información de cada componente de color. Se podrían hacer más experimentos en diferentes aplicaciones como segmentación o fusión de imágenes a color. Se podrían extraer otro tipo de información por cada componente como la Entropía o la Energía.
 


%Por lo tanto en $K^{TH}$, el contraste  entre las \'areas brillantes y oscuras de la imagen $f$ es mayor.

%En la actualidad existen varios operadores de contraste compuestos y multiescala a partir de la transformada top-hat \cite{bai2012toggle}. Como el objetivo de este trabajo no es crear un nuevo operador de contraste sino validar la estrategia de ordenamiento de color  usaremos la ecuaci\'on \ref{contraste} en la aplicaci\'on de mejoramiento de contraste. Con la misma se podr\'a comparar los diferentes trabajos del estado de arte y la propuesta en la secci\'on de experimentos.

%\section{Section title}
%\label{sec:1}
%Citation of \cite{trouiller2002drug}.
%\subsection{Subsection title}
%\label{sec:2}
%as required. Don't forget to give each section
%and subsection a unique label (see Sect.~\ref{sec:1}).
%\paragraph{Paragraph headings} Use paragraph headings as needed.
%\begin{equation}
%a^2+b^2=c^2
%\end{equation}

% For one-column wide figures use
%\begin{figure}
%\centering
% Use the relevant command to insert your figure file.
% For example, with the graphicx package use
%  \includegraphics{example.eps}
% figure caption is below the figure
%\caption{Please write your figure caption here}
%\label{fig:1}       % Give a unique label
%\end{figure}
%
% For two-column wide figures use
%\begin{figure}
%\centering
% Use the relevant command to insert your figure file.
% For example, with the graphicx package use
%  \includegraphics[width=0.75\textwidth]{example.eps}
% figure caption is below the figure
%\caption{Please write your figure caption here}
%\label{fig:2}       % Give a unique label
%\end{figure}
%
% For tables use
%\begin{table}[h]
% table caption is above the table
%\caption{Please write your table caption here}
%\centering
%\label{tab:1}       % Give a unique label
% For LaTeX tables use
%\begin{tabular}{lll}
%\hline\noalign{\smallskip}
%first & second & third  \\
%\noalign{\smallskip}\hline\noalign{\smallskip}
%number & number & number \\
%number & number & number \\
%\noalign{\smallskip}\hline
%\end{tabular}
%\end{table}


%\begin{acknowledgements}
%If you'd like to thank anyone, place your comments here
%and remove the percent signs.
%\end{acknowledgements}

% BibTeX users please use
\bibliographystyle{spbasic}
\bibliography{sample}   % name your BibTeX data base

% Non-BibTeX users please use
%\begin{thebibliography}{}
%
% and use \bibitem to create references. Consult the Instructions
% for authors for reference list style.
%
% Format for Journal Reference
%\bibitem[Author I(1999)]{RefJ}
%Author I (year) Article title. Journal Title-Abbreviated Vol: pp--pp
% Format for books
%\bibitem[Author and Smith(2001)]{RefB}
%Author I, Smith J (year) Book title. Publisher, Place, pp numbers
% etc
%\end{thebibliography}

\end{document}
% end of file template.tex
